%%%%%%%%%%%%%%%%%%%%%%%%%%%%%%%%%%%%%%%%%
% Masters/Doctoral Thesis
% LaTeX Template
% Version 2.5 (27/8/17)
%
% This template was downloaded from:
% http://www.LaTeXTemplates.com
%
% Version 2.x major modifications by:
% Vel (vel@latextemplates.com)
%
% This template is based on a template by:
% Steve Gunn (http://users.ecs.soton.ac.uk/srg/softwaretools/document/templates/)
% Sunil Patel (http://www.sunilpatel.co.uk/thesis-template/)
%
% Template license:
% CC BY-NC-SA 3.0 (http://creativecommons.org/licenses/by-nc-sa/3.0/)
%
%%%%%%%%%%%%%%%%%%%%%%%%%%%%%%%%%%%%%%%%%

%----------------------------------------------------------------------------------------
%	PACKAGES AND OTHER DOCUMENT CONFIGURATIONS
%----------------------------------------------------------------------------------------

\documentclass[
11pt, % The default document font size, options: 10pt, 11pt, 12pt
%oneside, % Two side (alternating margins) for binding by default, uncomment to switch to one side
english, % ngerman for German
singlespacing, % Single line spacing, alternatives: onehalfspacing or doublespacing
%draft, % Uncomment to enable draft mode (no pictures, no links, overfull hboxes indicated)
%nolistspacing, % If the document is onehalfspacing or doublespacing, uncomment this to set spacing in lists to single
%liststotoc, % Uncomment to add the list of figures/tables/etc to the table of contents
%toctotoc, % Uncomment to add the main table of contents to the table of contents
%parskip, % Uncomment to add space between paragraphs
%nohyperref, % Uncomment to not load the hyperref package
headsepline, % Uncomment to get a line under the header
%chapterinoneline, % Uncomment to place the chapter title next to the number on one line
%consistentlayout, % Uncomment to change the layout of the declaration, abstract and acknowledgements pages to match the default layout
]{MastersDoctoralThesis} % The class file specifying the document structure

\usepackage[utf8]{inputenc} % Required for inputting international characters
\usepackage[T1]{fontenc} % Output font encoding for international characters

\usepackage{mathpazo} % Use the Palatino font by default

% \usepackage[backend=bibtex,style=authoryear,natbib=true]{biblatex} % Use the bibtex backend with the authoryear citation style (which resembles APA)
\usepackage[backend=bibtex,style=numeric,natbib=true]{biblatex} % Use the bibtex backend with the authoryear citation style (which resembles APA)

\addbibresource{example.bib} % The filename of the bibliography
\addbibresource{ribes19bib.bib}
\addbibresource{other_references.bib}

\usepackage[autostyle=true]{csquotes} % Required to generate language-dependent quotes in the bibliography

\usepackage{mystyle}

%----------------------------------------------------------------------------------------
%	MARGIN SETTINGS
%----------------------------------------------------------------------------------------

\geometry{
	paper=a4paper, % Change to letterpaper for US letter
	inner=2.5cm, % Inner margin
	outer=3.8cm, % Outer margin
	bindingoffset=.5cm, % Binding offset
	top=1.5cm, % Top margin
	bottom=1.5cm, % Bottom margin
	%showframe, % Uncomment to show how the type block is set on the page
}

%----------------------------------------------------------------------------------------
%	THESIS INFORMATION
%----------------------------------------------------------------------------------------

\thesistitle{Using Random Fourier Features with Random Forest} % Your thesis title, this is used in the title and abstract, print it elsewhere with \ttitle
\supervisor{Lluís A. \textsc{Belanche}} % Your supervisor's name, this is used in the title page, print it elsewhere with \supname
\examiner{} % Your examiner's name, this is not currently used anywhere in the template, print it elsewhere with \examname
\degree{Bachelor Degree in Computer Science} % Your degree name, this is used in the title page and abstract, print it elsewhere with \degreename
\author{Albert \textsc{Ribes}} % Your name, this is used in the title page and abstract, print it elsewhere with \authorname
\addresses{} % Your address, this is not currently used anywhere in the template, print it elsewhere with \addressname

% \subject{Computing} % Your subject area, this is not currently used anywhere in the template, print it elsewhere with \subjectname
\subject{Major in Computing}
\keywords{} % Keywords for your thesis, this is not currently used anywhere in the template, print it elsewhere with \keywordnames
\university{\href{https://www.upc.edu/ca}{Universitat Politècnica de Catalunya (UPC) -- BarcelonaTech}} % Your university's name and URL, this is used in the title page and abstract, print it elsewhere with \univname
\department{\href{https://www.fib.upc.edu/en/research/departments/computer-science}{Computer Science}} % Your department's name and URL, this is used in the title page and abstract, print it elsewhere with \deptname
% \group{\href{http://researchgroup.university.com}{Research Group Name}} % Your research group's name and URL, this is used in the title page, print it elsewhere with \groupname
\faculty{\href{https://www.fib.upc.edu}{Facultat d’Informàtica de Barcelona (FIB)}} % Your faculty's name and URL, this is used in the title page and abstract, print it elsewhere with \facname

\AtBeginDocument{
\hypersetup{pdftitle=\ttitle} % Set the PDF's title to your title
\hypersetup{pdfauthor=\authorname} % Set the PDF's author to your name
\hypersetup{pdfkeywords=\keywordnames} % Set the PDF's keywords to your keywords
}

\begin{document}

\frontmatter % Use roman page numbering style (i, ii, iii, iv...) for the pre-content pages

\pagestyle{plain} % Default to the plain heading style until the thesis style is called for the body content

%----------------------------------------------------------------------------------------
%	TITLE PAGE
%----------------------------------------------------------------------------------------

\begin{titlepage}
\begin{center}

\vspace*{.06\textheight}
{\scshape\LARGE \univname\par}\vspace{1.5cm} % University name
% \textsc{\Large Doctoral Thesis}\\[0.5cm] % Thesis type
% \textsc{\Large Degree Final Project}\\[0.5cm] % Thesis type
\textsc{\Large Bachelor's Thesis}\\[0.5cm] % Thesis type

\HRule \\[0.4cm] % Horizontal line
{\huge \bfseries \ttitle\par}\vspace{0.4cm} % Thesis title
\HRule \\[1.5cm] % Horizontal line

\begin{minipage}[t]{0.4\textwidth}
\begin{flushleft} \large
\emph{Author:}\\
\authorname % Author name - remove the \href bracket to remove the link
\end{flushleft}
\end{minipage}
\begin{minipage}[t]{0.4\textwidth}
\begin{flushright} \large
\emph{Supervisor:} \\
\href{https://www.cs.upc.edu/~belanche/}{\supname} % Supervisor name - remove the \href bracket to remove the link
\end{flushright}
\end{minipage}\\[3cm]

\vfill

% \large \textit{A thesis submitted in fulfillment of the requirements\\ for the degree of \degreename}\\[0.3cm] % University requirement text
% \textit{in the}\\[0.4cm]
% \groupname\\\deptname\\[2cm] % Research group name and department name
\deptname\\[2cm] % Research group name and department name

\vfill

{\large \today}\\[4cm] % Date
%\includegraphics{Logo} % University/department logo - uncomment to place it

\vfill
\end{center}
\end{titlepage}

%----------------------------------------------------------------------------------------
%	DECLARATION PAGE
%----------------------------------------------------------------------------------------

% \begin{declaration}
% \addchaptertocentry{\authorshipname} % Add the declaration to the table of contents
% \noindent I, \authorname, declare that this thesis titled, \enquote{\ttitle} and the work presented in it are my own. I confirm that:
%
% \begin{itemize}
% \item This work was done wholly or mainly while in candidature for a research degree at this University.
% \item Where any part of this thesis has previously been submitted for a degree or any other qualification at this University or any other institution, this has been clearly stated.
% \item Where I have consulted the published work of others, this is always clearly attributed.
% \item Where I have quoted from the work of others, the source is always given. With the exception of such quotations, this thesis is entirely my own work.
% \item I have acknowledged all main sources of help.
% \item Where the thesis is based on work done by myself jointly with others, I have made clear exactly what was done by others and what I have contributed myself.\\
% \end{itemize}
%
% \noindent Signed:\\
% \rule[0.5em]{25em}{0.5pt} % This prints a line for the signature
%
% \noindent Date:\\
% \rule[0.5em]{25em}{0.5pt} % This prints a line to write the date
% \end{declaration}

\cleardoublepage

%----------------------------------------------------------------------------------------
%	QUOTATION PAGE
%----------------------------------------------------------------------------------------

% \vspace*{0.2\textheight}
%
% \noindent\enquote{\itshape Thanks to my solid academic training, today I can write hundreds of words on virtually any topic without possessing a shred of information, which is how I got a good job in journalism.}\bigbreak
%
% \hfill Dave Barry

%----------------------------------------------------------------------------------------
%	ABSTRACT PAGE
%----------------------------------------------------------------------------------------

\begin{abstract}
\addchaptertocentry{\abstractname} % Add the abstract to the table of contents
\begin{pre-delivery}
	Kernel methods are powerful tools to capture nonlinear patters behind the
	data. They perform an implicit mapping to a high (even infinite) dimensional
	feature space, but their running time increases significantly with the size
	of the dataset. In contrast, linear methods are much faster to train, but
	they have a more limited representational power. Fortunately, there are some
	ways to construct a random mapping into a relatively low-dimensional feature
	space which allows models to capture nonlinear patters with linear methods.
	In this project we study how Random Fourier Features and the \Nys\ method
	can be used with Logistic Regression, Support Vector Machine and Decision
	Tree to speed up both a single model and an ensemble of estimators. We
	empirically show that they can be used to successfully increase the
	accuracy of Logistic Regression and Support Vector Machine and we study
	different ways to train and ensemble with this estimators using Random
	Fourier Features and \Nys.

\end{pre-delivery}
\end{abstract}





%----------------------------------------------------------------------------------------
%	ABSTRACT PAGE
%----------------------------------------------------------------------------------------

\begin{abstract}
\addchaptertocentry{\abstractname} % Add the abstract to the table of contents
\begin{pre-delivery}
	Los métodos kernel son herramientas muy potentes para capturar patrones
	no lineales tras los datos. Para ello realizan una transformación implícita
	a un espacio de alta (incluso infinita) dimensionalidad, pero su tiempo de
	ejecución crece significativamente con el tamaño del conjunto de datos. Por
	contra, los métodos lineales son más rápidos de entrenar, pero tienen una
	capacidad de representación más limitada. Afortunadamente, hay formas de
	construir una transformación aleatoria a un espacio de baja dimensionalidad
	que permite a los modelos capturar patrones no lineales con métodos
	lineales. En este proyecto estudiamos como las Random Fourier Features y el
	método \Nys se pueden usar junto con Regresión Logística, Máquina de Vectores
	de Soporte y Árbol de Decisión para mejorar un modelo solo
	y también un conjunto de estimadores. Mostramos empíricamente que se pueden
	usar para aumentar con éxito la precisión de Regresión Logística y Máquina
	de Vectores de Soporte y estudiamos varias formas de entrenar un conjunto
	de estimadores usando Random Fourier Features y \Nys.
\end{pre-delivery}

\end{abstract}





%----------------------------------------------------------------------------------------
%	ABSTRACT PAGE
%----------------------------------------------------------------------------------------

\begin{abstract}
\addchaptertocentry{\abstractname} % Add the abstract to the table of contents

\begin{pre-delivery}
	Els mètodes kernel són eines molt potents per capturar patrons no lineals rere
	les dades. Per fer-ho fan una transformació implícita cap a un espai de
	alta (fins i tot infinita) dimensionalitat, però el seu temps d'execució
	augmenta significativament amb la grandaria del conjunt de dades. En
	contraposició, el mètodes lineals són més ràpids d'entrenar, però tenen una
	capacitat de representació més limitada. Afortunadament, hi ha maneres de
	construir una transformació aleatòria cap a un un espai de baixa
	dimensionalitat que permeti als models capturar patrons no lineals amb mètodes
	lineals. En aquest projecte estudiem com les Random Fourier Features y el
	mètode \Nys\ es poden fer servir juntament amb Regressió Logística, Màquina de
	Vectors de Suport i Arbre de Decisió per a millorar un sol model i també un
	conjunt d'estimadors. Mostrem empíricament que es poden fer servir per
	augmentar satisfactòriament la precisió de Regressió Logística i Màquina
	de Vectors de Suport i estudiem algunes maneres de entrenar un conjunt
	d'estimadors fent servir Random Fourier Featuers i \Nys.
\end{pre-delivery}

\end{abstract}



%----------------------------------------------------------------------------------------
%	ACKNOWLEDGEMENTS
%----------------------------------------------------------------------------------------

% \begin{acknowledgements}
% \addchaptertocentry{\acknowledgementname} % Add the acknowledgements to the table of contents
% The acknowledgments and the people to thank go here, don't forget to include your project advisor\ldots
% \end{acknowledgements}

%----------------------------------------------------------------------------------------
%	LIST OF CONTENTS/FIGURES/TABLES PAGES
%----------------------------------------------------------------------------------------

\tableofcontents % Prints the main table of contents

% \listoffigures % Prints the list of figures

% \listoftables % Prints the list of tables

%----------------------------------------------------------------------------------------
%	ABBREVIATIONS
%----------------------------------------------------------------------------------------

% \begin{abbreviations}{ll} % Include a list of abbreviations (a table of two columns)
%
% \textbf{ML} & \textbf{M}achine \textbf{L}earning\\
% \textbf{SVM} & \textbf{S}upport \textbf{V}ector \textbf{M}achine\\
% \textbf{RBF} & \textbf{R}adial \textbf{B}asis \textbf{F}unction\\
% \textbf{DT} & \textbf{D}ecision \textbf{T}ree\\
% \textbf{RF} & \textbf{R}andom \textbf{F}orest\\
% \textbf{RFF} & \textbf{R}andom \textbf{F}ourier \textbf{F}eatures\\
% \textbf{CV} & \textbf{C}ross-\textbf{V}alidation\\
%
%
% \end{abbreviations}

%----------------------------------------------------------------------------------------
%	PHYSICAL CONSTANTS/OTHER DEFINITIONS
%----------------------------------------------------------------------------------------

% \begin{constants}{lr@{${}={}$}l} % The list of physical constants is a three column table
%
% % The \SI{}{} command is provided by the siunitx package, see its documentation for instructions on how to use it
%
% Speed of Light & $c_{0}$ & \SI{2.99792458e8}{\meter\per\second} (exact)\\
% %Constant Name & $Symbol$ & $Constant Value$ with units\\
%
% \end{constants}

%----------------------------------------------------------------------------------------
%	SYMBOLS
%----------------------------------------------------------------------------------------

% \begin{symbols}{lll} % Include a list of Symbols (a three column table)
%
% $a$ & distance & \si{\meter} \\
% $P$ & power & \si{\watt} (\si{\joule\per\second}) \\
% %Symbol & Name & Unit \\
%
% \addlinespace % Gap to separate the Roman symbols from the Greek
%
% $\omega$ & angular frequency & \si{\radian} \\
%
% \end{symbols}

%----------------------------------------------------------------------------------------
%	DEDICATION
%----------------------------------------------------------------------------------------

% \dedicatory{For/Dedicated to/To my\ldots}

%----------------------------------------------------------------------------------------
%	THESIS CONTENT - CHAPTERS
%----------------------------------------------------------------------------------------

\mainmatter % Begin numeric (1,2,3...) page numbering

\pagestyle{thesis} % Return the page headers back to the "thesis" style

% Include the chapters of the thesis as separate files from the Chapters folder
% Uncomment the lines as you write the chapters

% \include{Chapters/Chapter1}
%\include{Chapters/Chapter2}
%\include{Chapters/Chapter3}
%\include{Chapters/Chapter4}
%\include{Chapters/Chapter5}

% Chapter Template

\chapter{Introduction} % Main chapter title

\label{Chapter1} % Change X to a consecutive number; for referencing this chapter elsewhere, use \ref{ChapterX}

% \usepackage{mystyle}

%----------------------------------------------------------------------------------------
%	SECTION 1
%----------------------------------------------------------------------------------------

\section{Problem to solve}

\begin{note}
  \begin{itemize}
    \item Trade-off between accuracy and train time is not good
  \end{itemize}
\end{note}



%----------------------------------------------------------------------------------------
%	SECTION 2
%----------------------------------------------------------------------------------------

\section{Why is it important?}

\begin{note}
  \begin{itemize}
    \item Avances en este campo permitirían usarlo en otras ciencias como medicina,
    economía, sociedad
    \item Muchas tareas que ahora tiene que hacer un humano podría hacerlas una
    máquina, ahorrando tiempo y dinero
  \end{itemize}
\end{note}


%----------------------------------------------------------------------------------------
%	SECTION 3
%----------------------------------------------------------------------------------------

\section{Project proposal}

\begin{note}
  \begin{itemize}
    \item Existe una batería de técnicas que son buenas, pero que nadie las
    ha combinado. Son:
    \begin{itemize}
      \item Modelos simples
      \item Ensembles
      \item kernel trick
      \item Aproximaciones de kernel
    \end{itemize}
    \item La propuesta es combinar todo esto para mejorar el trade-off
    \item Sostenemos las siguientes hipótesis:
    \begin{itemize}
      \item Se podría hacer un ensemble con modelos distintos a DT
      \item Se puede aproximar una RBF-SVM pero con el coste de una lineal
      \item RFF + Bootstrap quizá es demasiado aleatorio
      \item Los modelos que no se basan en productos escalares no se
      beneficiarán tanto de usar RFF
    \end{itemize}
    \item Lo que se hará en cada capítulo del trabajo
  \end{itemize}
\end{note}

% Chapter Template

\chapter{Background Information and Theory} % Main chapter title

\label{Chapter2} % Change X to a consecutive number; for referencing this chapter elsewhere, use \ref{ChapterX}

%----------------------------------------------------------------------------------------
%	SECTION 1
%----------------------------------------------------------------------------------------

\section{Machine Learning}
\begin{note}
  \begin{itemize}
    \item Clasificación y regresión
    \item Cross-validation
    \item Qué son los datos de train y test, y por qué se hace esa partición
    \item Qué es el sobre-ajuste
  \end{itemize}
\end{note}

\begin{note}
  \section{Review de los principales modelos que existen}
\end{note}
  \subsection{Decision Tree}
  \subsection{Logistic Regression}
  \subsection{Support Vector Machines}

\section{Ensemble Methods}
  \subsection{Bagging}
    \begin{note}
      \begin{itemize}
        \item Bootstrap
        \item Random Forest
      \end{itemize}
    \end{note}

\section{The kernel trick}
  \subsection{The RBF kernel}

\section{Random Fourier Features}
\section{Nystroem}

% Chapter Template

\chapter{Project Development} % Main chapter title

\label{Chapter3} % Change X to a consecutive number; for referencing this chapter elsewhere, use \ref{ChapterX}

%----------------------------------------------------------------------------------------
%	SECTION 1
%----------------------------------------------------------------------------------------

\section{General Idea}
\begin{note}
  \begin{itemize}
    \item Hemos visto que se puede sacar una aproximación aleatoria de la
    función implícita de un shift invariant kernel. Esto tiene 2 ventajas
    \begin{itemize}
      \item Podemos transformar los datos directamente
      \item Podemos producir pequeñas variaciones de un mismo dataset, todas
      ellas válidas
    \end{itemize}
    \item Las 4 tipos de modelos que he definido. Referencia a la foto
    \item ¿Por qué he cogido estos 4 modelos? ¿No podrían haber sido otros?
    ¿Que tienen estos de bueno? Me he inspirado en Random Forest
    \item Hay por ahí algún paper que compara RFF y \Nys
  \end{itemize}
\end{note}
\begin{note}
  \subsection{State of the art con las RFF}
\end{note}
\begin{note}
  \begin{itemize}
    \item Se ha trabajado poco con ellas. Solo he encontrado 2 usos:
    \begin{itemize}
      \item Stacked kernel network (referencia): usarlas junto a una red
      neuronal para tener más niveles de aprendizaje no lineal
      \item RFF with SVM (referencia): usar una SVM sin kernel con los datos
      mapeados usando RFF
    \end{itemize}
  \end{itemize}
\end{note}
\begin{note}
  \subsection{State of the art con las \Nys}
\end{note}


\section{Hyper-parameters}
\begin{note}
  \begin{itemize}
    \item Existen los siguientes:
    \begin{itemize}
      \item min-impurity-decrease para DT
      \item C para SVM
      \item gamma para RFF y \Nys
      \item cantidad de features para RFF y \Nys
      \item cantidad de estimadores para ensembles
    \end{itemize}
    \item Hemos usado los siguientes valores:
    \begin{itemize}
      \item Cantidad de features a 500
      \item Cantidad de estimadores a 50
      \item En modelos simples, el parámetro por crossvalidation
      \item En modelos simples con RFF, el parámetro por crossvalidation
      y una gamma que sobreajuste
      \item En modelos con ensemble, parámetros que sobreajusten y la gamma
      por crossvalidation
      \item En RBF-SVM, la gamma por gamest y el parámetro por crossvalidation
    \end{itemize}
  \end{itemize}
\end{note}
\section{Hypothesis}
\begin{note}
  \begin{enumerate}
    \item Podemos aproximar bien una RBF-SVM
    \item Puede tener sentido hacer ensembles con otros modelos a DT
    \item RFF + Bootstrap puede ser malo
    \item Si el modelo no se basa en productos escalares no se
    feneficiará tanto
  \end{enumerate}
\end{note}
\begin{note}
  \subsection{Planteamiento de los experimentos}
\end{note}
\begin{note}
  \begin{enumerate}
    \item Hipótesis: Aproximar RBF-SVM
    \begin{enumerate}
      \item Comparar una RBF-SVM con SVM normal que use RFF
    \end{enumerate}
    \item Hipótesis: Ensembles con otros
    \begin{enumerate}
      \item Logit normal vs. Logit con RFF
      \item Logit normal vs. Logit con RFF Black Bag
      \item Logit normal vs. Logit con RFF Grey Bag
      \item Logit normal vs. Logit con RFF Grey Ensemble
      \hrule
      \item Linear-SVM vs Linear-SVM con RFF
      \item Linear-SVM vs Linear-SVM con RFF Black Bag
      \item Linear-SVM vs Linear-SVM con RFF Grey Bag
      \item Linear-SVM vs Linear-SVM con RFF Grey Ensemble
    \end{enumerate}
    \item Hipótesis: RFF + Bootstrap
    \begin{enumerate}
      \item Logit con RFF Grey Bag vs Logit con RFF Grey Ensemble
      \item Logit con RFF Black Bag vs Logit con RFF Black Ensemble (los
      dos con un solo estimador)
      \hrule
      \item Linear-SVM con RFF Grey Bag vs Linear-SVM con RFF Grey Ensemble
      \item Linear-SVM con RFF Black Bag vs Linear-SVM con RFF Black Ensemble (los
    \end{enumerate}
    \item Hipótesis: DT + RFF
    \begin{itemize}
      \item DT vs DT con RFF
      \item DT vs DT con RFF Black Bag
      \item DT vs DT con RFF Black Ensemble
      \item DT vs DT con RFF Grey Bag
      \item DT vs DT con RFF Grey Ensemble
    \end{itemize}
  \end{enumerate}
\end{note}

\section{Datasets}
\begin{note}
  \begin{itemize}
    \item 8 Datasets
    \item Normalizados
    \item Únicamente tienen variables numéricas, no categóricas
    \item Únicamente problemas de clasificación
    \item Algunas cosas particulares que he hecho:
    \begin{itemize}
      \item Mezclar datos de train y de test para luego hacer mi propia
      separación
      \item Cuando había poca presencia de una clase, hacer un resampling para
      igualar las cantidades
      \item No trabajar cosas como el skiwness o los outliers
      \item Eliminar columnas en las que todo eran 0
      \item Reducir el conjunto de instancias
    \end{itemize}
  \end{itemize}
\end{note}

% Chapter Template

\chapter{Experimental Results} % Main chapter title

\label{Chapter4} % Change X to a consecutive number; for referencing this chapter elsewhere, use \ref{ChapterX}

%----------------------------------------------------------------------------------------
%	SECTION 1
%----------------------------------------------------------------------------------------

\begin{pre-delivery}
  In this chapter we will study the results obtained from the experiments and
  discuss the hypothesis that were considered for this project. All the results
  can be looked up in the \hyperref[Appendix1-1]{appendices}.

\section*{Hypothesis 1}
\label{disc:h1}

Hypothesis 1 claimed that Linear SVMs could achieve the accuracy of an
RBF-SVM but with much less training time. To check that we ran experiment
1.1, whose results can be seen in \ref{Appendix1-1}.

We can see that the models using Random Fourier Features or \Nys\ (shorted as
RFF and Nys.) reduce the error compared to a single Support Vector Machine
in the cases were using the real RBF kernel increased the
accuracy. In fact, the error incurred is always very close to the one of
RBF-SVM.

Regarding the training time, we observe that in most of the datasets it was
needed more time to train a single SVM using a random mapping than to train
an SVM with the RBF kernel. The only situations were the random mapping approach
saved us a lot of time is with datasets MNIST and Fashion MNIST.

We suggest that the reason why that happens is that the overhead of using
the random mapping is too big for small datasets compared to using the RBF
kernel, but becomes less relevant with bigger datasets, since the cost of
the random mapping is linear with the number of instances while the cost of the
optimising with the RBF kernel is cubic.

We conclude that the results obtained provide evidence to assert that the
hypothesis 1 is true for large datasets.

\section*{Hypothesis 2}
\label{disc:h2}

Hypothesis 2 suggested that using a random mapping to train an ensemble of
Logistic Regression models and Support Vector Machines could improve the
accuracy of a single model. To give support to this hypothesis we proposed
the experiments 2.1, 2.2, 2.3 and 2.4. The results of these experiments
can be looked up in
\ref{Appendix2-1},
\ref{Appendix2-2},
\ref{Appendix2-3} and
\ref{Appendix2-4}, respectively.

First, we wanted to see what was the effect of using the random mapping with
a single model, and then with an ensemble.

Experiments \hyperref[Appendix2-1]{2.1} and \hyperref[Appendix2-3]{2.2} were
defined for Logistic Regression.
Except for Fall Detection and Segment, we can see an increase of the accuracy
when using the random mapping on a single estimator. With most of the datasets
the increase is between 2 and 4 \%, and with Vowel it is about 26 \% of
increase.
When used with an ensemble, all of the Box Models show a very similar
improvement to a single estimator.
To see better the differences, they have been put together in a
\hyperref[2_2:aux]{separate chart}.
% In \ref{2_2:aux} we can easily
% compare them. Comparació \hyperref[2_2:aux]{aquí}

\hyperref[Appendix2-3]{2.3} and \hyperref[Appendix2-4]{2.4} are the equivalent
experiments for SVM.

With a single model, we can see there is a
meaningful improve in the accuracy
with all datasets except for Fall Detection. The improvements are from 2 \%
(with Digits and Segment) to 35 \% with Vowel.

With an Ensemble of SVMs the increases are more of less the same as with a
single estimator. The differences can be seen in a \hyperref[2_4:aux]{separate
chart}.

Although we can see there are some benefits of using an ensemble of these models,
the training time is a lot higher than with a single one. Given that the
difference between the accuracy of a single model using a random mapping and
an ensemble of them is so tiny, there are not evidences that training an ensemble
of Logistic Regression or SVM with random mapping is worth it. However, in
situations were that tiny difference is very important it could make sense
to use it, specially if one can afford it.
% with the datasets Digits, Pen Digits, and
% Vowel. With the rest of the dataset we can also see some improve, but it is not
% too much.

% Se pueden comparar mejor \hyperref[2_4:aux]{aquí}
\section*{Hypothesis 3}
\label{disc:h3}

For this hypothesis we expected that the ``Ensemble'' models would get
better results than the ``Bags'' given that Bootstrap with random mapping
could cause too much randomization. To check that we proposed experiments
3.1, 3.2, 3.3 and 3.4. The results can be seen in
\ref{Appendix3-1},
\ref{Appendix3-2},
\ref{Appendix3-3} and
\ref{Appendix3-4}.

In 3.1 we compare the White Box Models with Logistic Regression. It can be seen
that the results are very similar. We can just see a very litle improve in using
an Ensemble in Digits, Pen Digits and Vowel (with \Nys).

With the Black Models we can see an improve in more datasets.
Digits, MNIST, Pen Digits, Segment and Vowel present better results with the
Ensemble models.

Regarding the SVMs (in 3.3 and 3.4), the difference is only seen with Vowel,
either in the White and Black Models.

Based on the results, we can say that whether to perform a bootstrap or not
together with a random mapping doen't make much difference. However, seen
that Bootstrap does not benefit the models, it may be better to avoid using
it.

\section*{Hypothesis 4}
\label{disc:h4}
Hypothesis 4 claimed that Decision Tree could not benefit of using
Random Fourier Features or \Nys. To check that, we proposed experiments
4.1 and 4.2. The first one, which can be seen in \ref{Appendix4-1} was
to compare a single tree, and the second one, in \ref{Appendix4-2}, an
ensemble. For 4.2 a Random Forest was used instead of a Decision Tree since
they benefit with Bagging.

With a single model, we can see no improvement of using a random mapping for
any of the datasets. Besides, the training time is increased.

With ensembles, however, there is a very little improvement with
Pen Digits and Vowel. But as for the majority of the datasets the error is
increased, we can say that using RFF of \Nys\ with Decision Tree is in
general a bad choice.

We observe there are big differences in the training time of a Random Forest
and the Ensembles of Decision Tree with random mapping. We guess that the
reason for that is that Random Forest is a well known algorithm and it has
been highly optimised, whilst the Decision Tree have a more straightforward
procedure which is less efficient.

\end{pre-delivery}

% \begin{note}
%   \section{Enfrentar resultados 2 a 2}
% \end{note}
% \begin{note}
%   \section*{Exp 1-1}
%   \begin{itemize}
%     \item Nunca jamás conseguimos sacar mejores resultados que una RBF-SVM
%     \item Pero los resultados que conseguimos son bastante parecidos en algunos
%     casos
%     \item En el experimento que hamos hecho, los tiempos de los métodos lineales
%     son mucho mayores que los de la RBF-SVM
%     \item Pero hay que atribuirlo al pequeño tamaño del dataset que hemos usado.
%     En un dataset más grande sí se notaría
%     \item Efectivamente, he realizado el experimento en fashion-mnist, y no hay
%     comparación
%   \end{itemize}
%
%   \section*{Exp 2-1}
%   \begin{itemize}
%     \item En general sí conseguimos mejorar sustancialmente un logit
%     \item Pero los tiempos son un poco mayores
%     \item Con datasets mayores, es de esperar que los tiempos no se disparen
%     demasiado
%   \end{itemize}
%
% \section*{Exp 2-2}
% \begin{itemize}
%   \item En los mismos datasets en los que un solo logit no había conseguido hacer
%   nada, aquí tampoco han hecho nada
%   \item Entre logit normal y logit con ensemble, la precisión es más o menos la
%   misma, pero el tiempo es mucho mayor
%   \item En el caso logit, realmente puede deberse a que no hemos hecho nada
%   para sobreajustar más.
% \end{itemize}
%
% \section*{2-3}
% \begin{itemize}
%   \item Exactamente igual que el anterior
% \end{itemize}
%
% \section*{2-4}
% \begin{itemize}
%   \item Exactamente igual que el anterior
% \end{itemize}
%
% \section*{2-5}
% \begin{itemize}
%   \item Conseguimos mejorar sustancialmente la precisión de la mayoría de
%   datasets
%   \item Los tiempos son mayores, pero todavía son aceptables
%   \item Suponemos que en datasets más grandes los tiempo serían más parecidos
% \end{itemize}
%
% \section*{2-6}
% \begin{itemize}
%   \item El dataset que se resistía se sigue resistiendo cuando hacemos un
%   ensemble
%   \item Ahora los tiempo sí que són muchísimo más grandes, quizá no sale a cuenta
%   hacer ensemble
%   \item Para ver si sale a cuenta hacer ensemble, habrá que verlo más adelante
% \end{itemize}
%
% \section*{2-7}
% \begin{itemize}
%   \item Exactamente igual que el anterior
% \end{itemize}
%
% \section*{2-8}
% \begin{itemize}
%   \item Exactamente igual que el anterior
% \end{itemize}
%
% \section*{3-1}
% \begin{itemize}
%   \item No podemos decir que haya una ensemble que claramente sea mejor que los
%   demás
%   \item Da la impresión que en algunos casos el grey es mejor que los otros,
%   pero no es nada concluyente
% \end{itemize}
%
% \section*{3-2}
% \begin{itemize}
%   \item Da la impresión que los black ensembles son un poco mejor que los grey,
%   pero no es nada concluyente
% \end{itemize}
%
% \section*{3-3}
% \begin{itemize}
%   \item Los resultados no son nada concluyentes
% \end{itemize}
%
% \section*{3-4}
% \begin{itemize}
%   \item Exactamente igual que el anterior
% \end{itemize}
%
% \section*{4-1}
% \begin{itemize}
%   \item Usar rff con un solo árbol no es nada beneficioso
%   \item Tanto los errores como los tiempos son más altos
% \end{itemize}
%
% \section*{4-2}
% \begin{itemize}
%   \item En algunos casos ha mejorado sustancialmente, y en otros no
%   \item Pero estamos comparando un solo árbol con 50: claramente los 50
%   tendrían que ser siempre mejores, y ese no es el caso
%   \item Los tiempos no tienen ninguna comparación
%   \item En algunos casos, incluso empeora el usar un ensemble
% \end{itemize}
%
% \section*{4-3}
% \begin{itemize}
%   \item Exactamente igual que el anterior
% \end{itemize}
%
% \section*{4-4}
% \begin{itemize}
%   \item Exactamente igual que el anterior
% \end{itemize}
%
% \section*{4-5}
% \begin{itemize}
%   \item Exactamente igual que el anterior
% \end{itemize}
%
% \end{note}
% \begin{note}
%   \section{Contrastar hipótesis con resultados}
% \end{note}

% Chapter Template

\chapter{Conclusion and Future Directions} % Main chapter title

\label{Chapter5} % Change X to a consecutive number; for referencing this chapter elsewhere, use \ref{ChapterX}

%----------------------------------------------------------------------------------------
%	SECTION 1
%----------------------------------------------------------------------------------------

% \begin{note}
%   \begin{itemize}
%     \item Problemas de regresión
%     \item Aproximar otros kernels a RBF
%     \item Ver el comportamiento con problemas que no sean tan bonitos (con
%     missings, clases desbalanceadas, etc)
%     \item Otros tipos de ensembles, como el boosting
%   \end{itemize}
% \end{note}
%
% \begin{note}
%   \begin{itemize}
%   \item Todo lo que hemos hecho solo sirve con datasets muy grandes. Para
%   pequeños, en general salimos perdiendo
%   \item Hemos conseuido hacerle un boost a logit
%   \item Si usamos RFF, en general no sale nada a cuenta hacer un
%   ensemble. No se mejora demasiado
%   \item Sí que podemos aproximar una RBF-SVM con una lineal a coste lineal
%   \item No hemos observado ninguna diferencia significativa entre usar RFF
%   o \Nys\
% \end{itemize}
%
% En general los únicos éxitos de este trabajo son:
% \begin{itemize}
%   \item Ahora podemos hacer un ensemble de logit y de svm, que antes no se podría
%   \item también hemos conseguido mejorar un poco un solo logit y svm
%   \item Hemos aprendido que da igual el tipo de ensemble que cojamos
% \end{itemize}
% \end{note}


\begin{pre-delivery}
  We have studied many ways of using Random Fourier Features and \Nys\ method
  to improve the trade-off between accuracy and training time of some
  Machine Learning Methods.

  Regarding Support Vector Machines, we've seen that a single SVM using
  random features can match the accuracy of an SVM using the RBF Kernel, but
  it is only worth the time for datasets with a lot of instances. In some
  situations an ensemble of SVMs with random features can increase a little
  bit the accuracy, but the additional needed training time is too much, and
  may not be suitable for most of the situations.

  We've seen that Logistic Regression can also benefit from Random Fourier
  Features and \Nys, achieving better accuracy than a single one. Like with
  SVMs, Ensembles of Logistic Regression barely increase the accuracy and
  is much more expensive.

  We've also verified that a single Decision Tree does't benefit from using
  random features. For some problems an ensemble of Decision Trees using
  random features outperformed the Random Forest a little bit, but in others
  the accuracy was worse, so results are not conclusive.

  We've checked that using Bootstrap together with random features is not
  too much randomness for the models studied, and in fact maybe it should
  even be increased to benefit from them.

  Finally, we've not observed a clear difference in the performace of using
  Random Fourier Features compared to \Nys, so it seems that choosing one over
  the other doen't make a real difference.

  Here are some ideas to extend this work for future studies:
  \begin{itemize}
    \item This study was carried out only for classification problems. It may be
    interesting to run the same experiments regression problems.
    \item Using Random Fourier Features or \Nys\ to approximate other kernels
    than the RBF.
    \item Test these ideas with other types of ensembles, like Boosting.
    \item Study or characterise what problems can show a higher accuracy
    using an ensemble of Decision Trees with random features than
    using a Random Forest.
  \end{itemize}
\end{pre-delivery}

% Chapter Template

\chapter{Sustainability Report} % Main chapter title

\label{Chapter6} % Change X to a consecutive number; for referencing this chapter elsewhere, use \ref{ChapterX}

%----------------------------------------------------------------------------------------
%	SECTION 1
%----------------------------------------------------------------------------------------

\section{Environmental}
\section{Economic}
\section{Social}
  \subsection{Impacto Personal}
  \subsection{Impacto Social}
  \subsection{Riesgos Sociales}









%----------------------------------------------------------------------------------------
%	THESIS CONTENT - APPENDICES
%----------------------------------------------------------------------------------------

\appendix % Cue to tell LaTeX that the following "chapters" are Appendices

% Include the appendices of the thesis as separate files from the Appendices folder
% Uncomment the lines as you write the Appendices

% \include{Appendices/AppendixA}
%\include{Appendices/AppendixB}
%\include{Appendices/AppendixC}

% \label{sec:appendices}

% \include{Appendices/TemporalAppendix1-1}
% Appendix Template

\chapter{Results of experiment 1.1} % Main appendix title

\label{Appendix1-1} % Change X to a consecutive letter; for referencing this appendix elsewhere, use \ref{AppendixX}


\begin{figure}[ht]
  \centering
  \begin{subfigure}[b]{0.5\linewidth}
    \centering\includegraphics[width=\imgscale\linewidth]{Figures/1_1/covertype}
    \caption{prueba covertype}
    \label{fig:1_1_covertype}
  \end{subfigure}%
  \begin{subfigure}[b]{0.5\linewidth}
    \centering\includegraphics[width=\imgscale\linewidth]{Figures/1_1/digits}
    \caption{prueba digits}
    \label{fig:1_1_digits}
  \end{subfigure}
\end{figure}


\begin{figure}[ht]
  \centering
  \begin{subfigure}[b]{0.5\linewidth}
    \centering\includegraphics[width=\imgscale\linewidth]{Figures/1_1/fall_detection}
    \caption{prueba fall-detection}
    \label{fig:1_1_fall_detection}
  \end{subfigure}%
  \begin{subfigure}[b]{0.5\linewidth}
    \centering\includegraphics[width=\imgscale\linewidth]{Figures/1_1/mnist}
    \caption{prueba mnist}
    \label{fig:1_1_mnist}
  \end{subfigure}
\end{figure}


\begin{figure}[ht]
  \centering
  \begin{subfigure}[b]{0.5\linewidth}
    \centering\includegraphics[width=\imgscale\linewidth]{Figures/1_1/pen_digits}
    \caption{prueba pen-digits}
    \label{fig:1_1_pen_digits}
  \end{subfigure}%
  \begin{subfigure}[b]{0.5\linewidth}
    \centering\includegraphics[width=\imgscale\linewidth]{Figures/1_1/satellite}
    \caption{prueba satellite}
    \label{fig:1_1_satellite}
  \end{subfigure}
\end{figure}

\begin{figure}[ht]
  \centering
  \begin{subfigure}[b]{0.5\linewidth}
    \centering\includegraphics[width=\imgscale\linewidth]{Figures/1_1/segment}
    \caption{prueba segment}
    \label{fig:1_1_segment}
  \end{subfigure}%
  \begin{subfigure}[b]{0.5\linewidth}
    \centering\includegraphics[width=\imgscale\linewidth]{Figures/1_1/vowel}
    \caption{prueba vowel}
    \label{fig:1_1_vowel}
  \end{subfigure}
\end{figure}

% Appendix Template

\chapter{Results of experiment 2.1} % Main appendix title

\label{Appendix2-1} % Change X to a consecutive letter; for referencing this appendix elsewhere, use \ref{AppendixX}


\begin{figure}[ht]
  \centering
  \begin{subfigure}[b]{0.5\linewidth}
    \centering\captionsetup{width=.8\linewidth}\includegraphics[width=\imgscale\linewidth]{Figures/2_1/covertype}
    \caption{prueba covertype}
    \label{fig:2_1_covertype}
  \end{subfigure}%
  \begin{subfigure}[b]{0.5\linewidth}
    \centering\captionsetup{width=.8\linewidth}\includegraphics[width=\imgscale\linewidth]{Figures/2_1/digits}
    \caption{prueba digits}
    \label{fig:2_1_digits}
  \end{subfigure}
\end{figure}

\begin{figure}[ht]
  \centering
  \begin{subfigure}[b]{0.5\linewidth}
    \centering\captionsetup{width=.8\linewidth}\includegraphics[width=\imgscale\linewidth]{Figures/2_1/fall_detection}
    \caption{prueba fall-detection}
    \label{fig:2_1_fall_detection}
  \end{subfigure}%
  \begin{subfigure}[b]{0.5\linewidth}
    \centering\captionsetup{width=.8\linewidth}\includegraphics[width=\imgscale\linewidth]{Figures/2_1/mnist}
    \caption{prueba mnist}
    \label{fig:2_1_mnist}
  \end{subfigure}
\end{figure}


\begin{figure}[ht]
  \centering
  \begin{subfigure}[b]{0.5\linewidth}
    \centering\captionsetup{width=.8\linewidth}\includegraphics[width=\imgscale\linewidth]{Figures/2_1/pen_digits}
    \caption{prueba pen-digits}
    \label{fig:2_1_pen_digits}
  \end{subfigure}%
  \begin{subfigure}[b]{0.5\linewidth}
    \centering\captionsetup{width=.8\linewidth}\includegraphics[width=\imgscale\linewidth]{Figures/2_1/satellite}
    \caption{prueba satellite}
    \label{fig:2_1_satellite}
  \end{subfigure}
\end{figure}

\begin{figure}[ht]
  \centering
  \begin{subfigure}[b]{0.5\linewidth}
    \centering\captionsetup{width=.8\linewidth}\includegraphics[width=\imgscale\linewidth]{Figures/2_1/segment}
    \caption{prueba segment}
    \label{fig:2_1_segment}
  \end{subfigure}%
  \begin{subfigure}[b]{0.5\linewidth}
    \centering\captionsetup{width=.8\linewidth}\includegraphics[width=\imgscale\linewidth]{Figures/2_1/vowel}
    \caption{prueba vowel}
    \label{fig:2_1_vowel}
  \end{subfigure}
\end{figure}

% Appendix Template

\chapter{Results of experiment 2.2} % Main appendix title

\label{Appendix2-2} % Change X to a consecutive letter; for referencing this appendix elsewhere, use \ref{AppendixX}

\begin{figure}[ht]
  \centering
  \begin{subfigure}[b]{0.5\linewidth}
    \centering\captionsetup{width=.8\linewidth}\includegraphics[width=\imgscale\linewidth]{Figures/2_2/covertype}
    \caption{prueba covertype}
    \label{fig:2_2_covertype}
  \end{subfigure}%
  \begin{subfigure}[b]{0.5\linewidth}
    \centering\captionsetup{width=.8\linewidth}\includegraphics[width=\imgscale\linewidth]{Figures/2_2/digits}
    \caption{prueba digits}
    \label{fig:2_2_digits}
  \end{subfigure}
\end{figure}


\begin{figure}[ht]
  \centering
  \begin{subfigure}[b]{0.5\linewidth}
    \centering\captionsetup{width=.8\linewidth}\includegraphics[width=\imgscale\linewidth]{Figures/2_2/fall_detection}
    \caption{prueba fall-detection}
    \label{fig:2_2_fall_detection}
  \end{subfigure}%
  \begin{subfigure}[b]{0.5\linewidth}
    \centering\captionsetup{width=.8\linewidth}\includegraphics[width=\imgscale\linewidth]{Figures/2_2/mnist}
    \caption{prueba mnist}
    \label{fig:2_2_mnist}
  \end{subfigure}
\end{figure}


\begin{figure}[ht]
  \centering
  \begin{subfigure}[b]{0.5\linewidth}
    \centering\captionsetup{width=.8\linewidth}\includegraphics[width=\imgscale\linewidth]{Figures/2_2/pen_digits}
    \caption{prueba pen-digits}
    \label{fig:2_2_pen_digits}
  \end{subfigure}%
  \begin{subfigure}[b]{0.5\linewidth}
    \centering\captionsetup{width=.8\linewidth}\includegraphics[width=\imgscale\linewidth]{Figures/2_2/satellite}
    \caption{prueba satellite}
    \label{fig:2_2_satellite}
  \end{subfigure}
\end{figure}

\begin{figure}[ht]
  \centering
  \begin{subfigure}[b]{0.5\linewidth}
    \centering\captionsetup{width=.8\linewidth}\includegraphics[width=\imgscale\linewidth]{Figures/2_2/segment}
    \caption{prueba segment}
    \label{fig:2_2_segment}
  \end{subfigure}%
  \begin{subfigure}[b]{0.5\linewidth}
    \centering\captionsetup{width=.8\linewidth}\includegraphics[width=\imgscale\linewidth]{Figures/2_2/vowel}
    \caption{prueba vowel}
    \label{fig:2_2_vowel}
  \end{subfigure}
\end{figure}

% Appendix Template

\newcommand{\major}{2}
\newcommand{\minor}{3}

\newcommand{\undPrefix}{\major_\minor}
\newcommand{\dotPrefix}{\major.\minor}
\newcommand{\scoPrefix}{\major-\minor}
\newcommand{\filePrefix}{\undPrefix}

\chapter{Results of experiment \dotPrefix} % Main appendix title


\label{Appendix\scoPrefix} % Change X to a consecutive letter; for referencing this appendix elsewhere, use \ref{AppendixX}
These experiments are discussed \hyperref[disc:h2]{here}

\begin{figure}[ht]
  \centering
  \begin{subfigure}[t]{0.5\linewidth}
    \centering\captionsetup{width=.8\linewidth}\includegraphics[width=\imgscale\linewidth]{Figures/\filePrefix/covertype}
    \caption{Exp 2.3 with Covertype. Error is decreased by 10\% approx.}
    \label{fig:\undPrefix_covertype}
  \end{subfigure}%
  \begin{subfigure}[t]{0.5\linewidth}
    \centering\captionsetup{width=.8\linewidth}\includegraphics[width=\imgscale\linewidth]{Figures/\filePrefix/digits}
    \caption{Exp 2.3 with Digits. Error is decreased by 2\% approx.}
    \label{fig:\undPrefix_digits}
  \end{subfigure}
\end{figure}


\begin{figure}[ht]
  \centering
  \begin{subfigure}[t]{0.5\linewidth}
    \centering\captionsetup{width=.8\linewidth}\includegraphics[width=\imgscale\linewidth]{Figures/\filePrefix/fall_detection}
    \caption{Exp 2.3 with Fall Detection. Error is not decreased.}
    \label{fig:\undPrefix_fall_detection}
  \end{subfigure}%
  \begin{subfigure}[t]{0.5\linewidth}
    \centering\captionsetup{width=.8\linewidth}\includegraphics[width=\imgscale\linewidth]{Figures/\filePrefix/mnist}
    \caption{Exp 2.3 with MNIST. Error is decreased by 18\% approx.}
    \label{fig:\undPrefix_mnist}
  \end{subfigure}
\end{figure}


\begin{figure}[ht]
  \centering
  \begin{subfigure}[t]{0.5\linewidth}
    \centering\captionsetup{width=.8\linewidth}\includegraphics[width=\imgscale\linewidth]{Figures/\filePrefix/pen_digits}
    \caption{Exp 2.3 with Pen Digits. Error is decreased by 5\% approx.}
    \label{fig:\undPrefix_pen_digits}
  \end{subfigure}%
  \begin{subfigure}[t]{0.5\linewidth}
    \centering\captionsetup{width=.8\linewidth}\includegraphics[width=\imgscale\linewidth]{Figures/\filePrefix/satellite}
    \caption{Exp 2.3 with Satellite. Error is decreased by 7\% approx.}
    \label{fig:\undPrefix_satellite}
  \end{subfigure}
\end{figure}

\begin{figure}[ht]
  \centering
  \begin{subfigure}[t]{0.5\linewidth}
    \centering\captionsetup{width=.8\linewidth}\includegraphics[width=\imgscale\linewidth]{Figures/\filePrefix/segment}
    \caption{Exp 2.3 with Segment. Error is decreased by 2\% approx.}
    \label{fig:\undPrefix_segment}
  \end{subfigure}%
  \begin{subfigure}[t]{0.5\linewidth}
    \centering\captionsetup{width=.8\linewidth}\includegraphics[width=\imgscale\linewidth]{Figures/\filePrefix/vowel}
    \caption{Exp 2.3 with Vowel. Error is decreased by 35\% approx.}
    \label{fig:\undPrefix_vowel}
  \end{subfigure}
\end{figure}



\begin{figure}[ht]
  \centering
  \begin{subfigure}[t]{0.5\linewidth}
    \centering\captionsetup{width=.8\linewidth}\includegraphics[width=\imgscale\linewidth]{Figures/\filePrefix/fashion_mnist}
    \caption{Exp 2.3 with Fashion MNIST. Error is decreased by 2\% approx.}
    \label{fig:\undPrefix_segment}
  \end{subfigure}%
\end{figure}


\let\major\undefined
\let\minor\undefined

\let\undPrefix\undefined
\let\dotPrefix\undefined
\let\scoPrefix\undefined

\let\filePrefix\undefined

% Appendix Template

\chapter{Results of experiment 2.4} % Main appendix title

\label{Appendix2-4} % Change X to a consecutive letter; for referencing this appendix elsewhere, use \ref{AppendixX}

\begin{figure}[ht]
  \centering
  \begin{subfigure}[b]{0.5\linewidth}
    \centering\captionsetup{width=.8\linewidth}\includegraphics[width=\imgscale\linewidth]{Figures/2_4/covertype}
    \caption{prueba covertype}
    \label{fig:2_4_covertype}
  \end{subfigure}%
  \begin{subfigure}[b]{0.5\linewidth}
    \centering\captionsetup{width=.8\linewidth}\includegraphics[width=\imgscale\linewidth]{Figures/2_4/digits}
    \caption{prueba digits}
    \label{fig:2_4_digits}
  \end{subfigure}
\end{figure}


\begin{figure}[ht]
  \centering
  \begin{subfigure}[b]{0.5\linewidth}
    \centering\captionsetup{width=.8\linewidth}\includegraphics[width=\imgscale\linewidth]{Figures/2_4/fall_detection}
    \caption{prueba fall-detection}
    \label{fig:2_4_fall_detection}
  \end{subfigure}%
  \begin{subfigure}[b]{0.5\linewidth}
    \centering\captionsetup{width=.8\linewidth}\includegraphics[width=\imgscale\linewidth]{Figures/2_4/mnist}
    \caption{prueba mnist}
    \label{fig:2_4_mnist}
  \end{subfigure}
\end{figure}


\begin{figure}[ht]
  \centering
  \begin{subfigure}[b]{0.5\linewidth}
    \centering\captionsetup{width=.8\linewidth}\includegraphics[width=\imgscale\linewidth]{Figures/2_4/pen_digits}
    \caption{prueba pen-digits}
    \label{fig:2_4_pen_digits}
  \end{subfigure}%
  \begin{subfigure}[b]{0.5\linewidth}
    \centering\captionsetup{width=.8\linewidth}\includegraphics[width=\imgscale\linewidth]{Figures/2_4/satellite}
    \caption{prueba satellite}
    \label{fig:2_4_satellite}
  \end{subfigure}
\end{figure}

\begin{figure}[ht]
  \centering
  \begin{subfigure}[b]{0.5\linewidth}
    \centering\captionsetup{width=.8\linewidth}\includegraphics[width=\imgscale\linewidth]{Figures/2_4/segment}
    \caption{prueba segment}
    \label{fig:2_4_segment}
  \end{subfigure}%
  \begin{subfigure}[b]{0.5\linewidth}
    \centering\captionsetup{width=.8\linewidth}\includegraphics[width=\imgscale\linewidth]{Figures/2_4/vowel}
    \caption{prueba vowel}
    \label{fig:2_4_vowel}
  \end{subfigure}
\end{figure}


% Appendix Template

\newcommand{\major}{3}
\newcommand{\minor}{1}

\newcommand{\undPrefix}{\major_\minor}
\newcommand{\dotPrefix}{\major.\minor}
\newcommand{\scoPrefix}{\major-\minor}
\newcommand{\filePrefix}{\undPrefix}

% \chapter{Results of experiment 1.1} % Main appendix title
\chapter{Results of experiment \dotPrefix} % Main appendix title

% \label{Appendix1-1} % Change X to a consecutive letter; for referencing this appendix elsewhere, use \ref{AppendixX}

\label{Appendix\scoPrefix} % Change X to a consecutive letter; for referencing this appendix elsewhere, use \ref{AppendixX}

These experiments are discussed \hyperref[disc:h3]{here}
\begin{figure}[ht]
  \centering
  \begin{subfigure}[t]{0.5\linewidth}
    % \centering\captionsetup{width=.8\linewidth}\includegraphics[width=\imgscale\linewidth]{Figures/1_1/covertype}
    \centering\captionsetup{width=.8\linewidth}\includegraphics[width=\imgscale\linewidth]{Figures/\filePrefix/covertype}
    \caption{Exp. 3.1 with Covertype. White Box Models with Logistic Regression. There is not much difference between Box Models}
    \label{fig:\undPrefix_covertype}
  \end{subfigure}%
  \begin{subfigure}[t]{0.5\linewidth}
    \centering\captionsetup{width=.8\linewidth}\includegraphics[width=\imgscale\linewidth]{Figures/\filePrefix/digits}
    \caption{Exp. 3.1 with Digits. White Box Models with Logistic Regression. There is not much difference between Box Models}
    % \label{fig:1_1_digits}
    \label{fig:\undPrefix_digits}
  \end{subfigure}
\end{figure}


\begin{figure}[ht]
  \centering
  \begin{subfigure}[t]{0.5\linewidth}
    \centering\captionsetup{width=.8\linewidth}\includegraphics[width=\imgscale\linewidth]{Figures/\filePrefix/fall_detection}
    \caption{Exp. 3.1 with Fall Detection. White Box Models with Logistic Regression. There is not much difference between Box Models}
    \label{fig:\undPrefix_fall_detection}
  \end{subfigure}%
  \begin{subfigure}[t]{0.5\linewidth}
    \centering\captionsetup{width=.8\linewidth}\includegraphics[width=\imgscale\linewidth]{Figures/\filePrefix/mnist}
    \caption{Exp. 3.1 with MNIST. White Box Models with Logistic Regression. There is not much difference between Box Models}
    \label{fig:\undPrefix_mnist}
  \end{subfigure}
\end{figure}


\begin{figure}[ht]
  \centering
  \begin{subfigure}[t]{0.5\linewidth}
    \centering\captionsetup{width=.8\linewidth}\includegraphics[width=\imgscale\linewidth]{Figures/\filePrefix/pen_digits}
    \caption{Exp. 3.1 with Pen Digits. White Box Models with Logistic Regression. There is not much difference between Box Models}
    \label{fig:\undPrefix_pen_digits}
  \end{subfigure}%
  \begin{subfigure}[t]{0.5\linewidth}
    \centering\captionsetup{width=.8\linewidth}\includegraphics[width=\imgscale\linewidth]{Figures/\filePrefix/satellite}
    \caption{Exp. 3.1 with Satelite. White Box Models with Logistic Regression. There is not much difference between Box Models}
    \label{fig:\undPrefix_satellite}
  \end{subfigure}
\end{figure}

\begin{figure}[ht]
  \centering
  \begin{subfigure}[t]{0.5\linewidth}
    \centering\captionsetup{width=.8\linewidth}\includegraphics[width=\imgscale\linewidth]{Figures/\filePrefix/segment}
    \caption{Exp. 3.1 with Segment. White Box Models with Logistic Regression. There is not much difference between Box Models}
    \label{fig:\undPrefix_segment}
  \end{subfigure}%
  \begin{subfigure}[t]{0.5\linewidth}
    \centering\captionsetup{width=.8\linewidth}\includegraphics[width=\imgscale\linewidth]{Figures/\filePrefix/vowel}
    \caption{Exp. 3.1 with Vowel. White Box Models with Logistic Regression. There is not much difference between Box Models}
    \label{fig:\undPrefix_vowel}
  \end{subfigure}
\end{figure}


\let\major\undefined
\let\minor\undefined

\let\undPrefix\undefined
\let\dotPrefix\undefined
\let\scoPrefix\undefined

\let\filePrefix\undefined

% Appendix Template

\newcommand{\major}{3}
\newcommand{\minor}{2}

\newcommand{\undPrefix}{\major_\minor}
\newcommand{\dotPrefix}{\major.\minor}
\newcommand{\scoPrefix}{\major-\minor}
\newcommand{\filePrefix}{\undPrefix}

% \chapter{Results of experiment 1.1} % Main appendix title
\chapter{Results of experiment \dotPrefix} % Main appendix title

% \label{Appendix1-1} % Change X to a consecutive letter; for referencing this appendix elsewhere, use \ref{AppendixX}

\label{Appendix\scoPrefix} % Change X to a consecutive letter; for referencing this appendix elsewhere, use \ref{AppendixX}


\begin{figure}[ht]
  \centering
  \begin{subfigure}[b]{0.5\linewidth}
    % \centering\includegraphics[width=\imgscale\linewidth]{Figures/1_1/covertype}
    \centering\includegraphics[width=\imgscale\linewidth]{Figures/\filePrefix/covertype}
    \caption{prueba covertype}
    \label{fig:\undPrefix_covertype}
  \end{subfigure}%
  \begin{subfigure}[b]{0.5\linewidth}
    \centering\includegraphics[width=\imgscale\linewidth]{Figures/\filePrefix/digits}
    \caption{prueba digits}
    % \label{fig:1_1_digits}
    \label{fig:\undPrefix_digits}
  \end{subfigure}
\end{figure}


\begin{figure}[ht]
  \centering
  \begin{subfigure}[b]{0.5\linewidth}
    \centering\includegraphics[width=\imgscale\linewidth]{Figures/\filePrefix/fall_detection}
    \caption{prueba fall-detection}
    \label{fig:\undPrefix_fall_detection}
  \end{subfigure}%
  \begin{subfigure}[b]{0.5\linewidth}
    \centering\includegraphics[width=\imgscale\linewidth]{Figures/\filePrefix/mnist}
    \caption{prueba mnist}
    \label{fig:\undPrefix_mnist}
  \end{subfigure}
\end{figure}


\begin{figure}[ht]
  \centering
  \begin{subfigure}[b]{0.5\linewidth}
    \centering\includegraphics[width=\imgscale\linewidth]{Figures/\filePrefix/pen_digits}
    \caption{prueba pen-digits}
    \label{fig:\undPrefix_pen_digits}
  \end{subfigure}%
  \begin{subfigure}[b]{0.5\linewidth}
    \centering\includegraphics[width=\imgscale\linewidth]{Figures/\filePrefix/satellite}
    \caption{prueba satellite}
    \label{fig:\undPrefix_satellite}
  \end{subfigure}
\end{figure}

\begin{figure}[ht]
  \centering
  \begin{subfigure}[b]{0.5\linewidth}
    \centering\includegraphics[width=\imgscale\linewidth]{Figures/\filePrefix/segment}
    \caption{prueba segment}
    \label{fig:\undPrefix_segment}
  \end{subfigure}%
  \begin{subfigure}[b]{0.5\linewidth}
    \centering\includegraphics[width=\imgscale\linewidth]{Figures/\filePrefix/vowel}
    \caption{prueba vowel}
    \label{fig:\undPrefix_vowel}
  \end{subfigure}
\end{figure}


\let\major\undefined
\let\minor\undefined

\let\undPrefix\undefined
\let\dotPrefix\undefined
\let\scoPrefix\undefined

\let\filePrefix\undefined

\include{Appendices/Appendix3-3}
\include{Appendices/Appendix3-4}

% Appendix Template

\newcommand{\major}{4}
\newcommand{\minor}{1}

\newcommand{\undPrefix}{\major_\minor}
\newcommand{\dotPrefix}{\major.\minor}
\newcommand{\scoPrefix}{\major-\minor}
\newcommand{\filePrefix}{\undPrefix}

% \chapter{Results of experiment 1.1} % Main appendix title
\chapter{Results of experiment \dotPrefix} % Main appendix title

% \label{Appendix1-1} % Change X to a consecutive letter; for referencing this appendix elsewhere, use \ref{AppendixX}

\label{Appendix\scoPrefix} % Change X to a consecutive letter; for referencing this appendix elsewhere, use \ref{AppendixX}

These experiments are discussed \hyperref[disc:h4]{here}
\begin{figure}[ht]
  \centering
  \begin{subfigure}[t]{0.5\linewidth}
    % \centering\captionsetup{width=.8\linewidth}\includegraphics[width=\imgscale\linewidth]{Figures/1_1/covertype}
    \centering\captionsetup{width=.8\linewidth}\includegraphics[width=\imgscale\linewidth]{Figures/\filePrefix/covertype}
    \caption{Exp. 4.1 with Covertype. Random Samplers increase the error on Decision Tree.}
    \label{fig:\undPrefix_covertype}
  \end{subfigure}%
  \begin{subfigure}[t]{0.5\linewidth}
    \centering\captionsetup{width=.8\linewidth}\includegraphics[width=\imgscale\linewidth]{Figures/\filePrefix/digits}
    \caption{Exp. 4.1 with Digits. Random Samplers increase the error on Decision Tree.}
    % \label{fig:1_1_digits}
    \label{fig:\undPrefix_digits}
  \end{subfigure}
\end{figure}


\begin{figure}[ht]
  \centering
  \begin{subfigure}[t]{0.5\linewidth}
    \centering\captionsetup{width=.8\linewidth}\includegraphics[width=\imgscale\linewidth]{Figures/\filePrefix/fall_detection}
    \caption{Exp. 4.1 with Fall Detection. Random Samplers increase the error on Decision Tree.}
    \label{fig:\undPrefix_fall_detection}
  \end{subfigure}%
  \begin{subfigure}[t]{0.5\linewidth}
    \centering\captionsetup{width=.8\linewidth}\includegraphics[width=\imgscale\linewidth]{Figures/\filePrefix/mnist}
    \caption{Exp. 4.1 with MNIST. Random Samplers increase the error on Decision Tree.}
    \label{fig:\undPrefix_mnist}
  \end{subfigure}
\end{figure}


\begin{figure}[ht]
  \centering
  \begin{subfigure}[t]{0.5\linewidth}
    \centering\captionsetup{width=.8\linewidth}\includegraphics[width=\imgscale\linewidth]{Figures/\filePrefix/pen_digits}
    \caption{Exp. 4.1 with Pen Digits. Random Samplers increase the error on Decision Tree.}
    \label{fig:\undPrefix_pen_digits}
  \end{subfigure}%
  \begin{subfigure}[t]{0.5\linewidth}
    \centering\captionsetup{width=.8\linewidth}\includegraphics[width=\imgscale\linewidth]{Figures/\filePrefix/satellite}
    \caption{Exp. 4.1 with Satellite. Random Samplers make no difference on Decision Tree.}
    \label{fig:\undPrefix_satellite}
  \end{subfigure}
\end{figure}

\begin{figure}[ht]
  \centering
  \begin{subfigure}[t]{0.5\linewidth}
    \centering\captionsetup{width=.8\linewidth}\includegraphics[width=\imgscale\linewidth]{Figures/\filePrefix/segment}
    \caption{Exp. 4.1 with Segment. Random Samplers increase the error on Decision Tree.}
    \label{fig:\undPrefix_segment}
  \end{subfigure}%
  \begin{subfigure}[t]{0.5\linewidth}
    \centering\captionsetup{width=.8\linewidth}\includegraphics[width=\imgscale\linewidth]{Figures/\filePrefix/vowel}
    \caption{Exp. 4.1 with Vowel. Random Samplers increase the error on Decision Tree.}
    \label{fig:\undPrefix_vowel}
  \end{subfigure}
\end{figure}


\begin{figure}[ht]
  \centering
  \begin{subfigure}[t]{0.5\linewidth}
    \centering\captionsetup{width=.8\linewidth}\includegraphics[width=\imgscale\linewidth]{Figures/\filePrefix/fashion_mnist}
    \caption{Exp. 4.1 with Fashion MNIST. Random Samplers increase the error on Decision Tree.}
    \label{fig:\undPrefix_segment}
  \end{subfigure}%
\end{figure}


\let\major\undefined
\let\minor\undefined

\let\undPrefix\undefined
\let\dotPrefix\undefined
\let\scoPrefix\undefined

\let\filePrefix\undefined

% Appendix Template

\newcommand{\major}{4}
\newcommand{\minor}{2}

\newcommand{\undPrefix}{\major_\minor}
\newcommand{\dotPrefix}{\major.\minor}
\newcommand{\scoPrefix}{\major-\minor}
\newcommand{\filePrefix}{\undPrefix/rff}

\chapter{Results of experiment \dotPrefix} % Main appendix title


\label{Appendix\scoPrefix} % Change X to a consecutive letter; for referencing this appendix elsewhere, use \ref{AppendixX}

These experiments are discussed \hyperref[disc:h4]{here}


%%%%%%%%%%%%%%%%%%%%%%%%%%%%%%%%%%%%%%%%%%%%%%
%%%%%%%%%%%%%%%%%%%%%%%%%%%%%%%%%%%%%%%%%%%%%%
%%%%%%%%%%%%%%%%%%%%%%%%%%%%%%%%%%%%%%%%%%%%%%
%% Empieza lo nuevo
%%%%%%%%%%%%%%%%%%%%%%%%%%%%%%%%%%%%%%%%%%%%%%
%%%%%%%%%%%%%%%%%%%%%%%%%%%%%%%%%%%%%%%%%%%%%%
%%%%%%%%%%%%%%%%%%%%%%%%%%%%%%%%%%%%%%%%%%%%%%

\begin{figure}[H]
  \centering
  \renewcommand{\filePrefix}{\undPrefix/rff}
  \begin{subfigure}[t]{0.5\linewidth}
    \centering\captionsetup{width=.8\linewidth}\includegraphics[width=\imgscale\linewidth]{Figures/\filePrefix/covertype}
    % \caption{Exp. 4.2 with Covertype. Random Forest outperforms Ensembles of Decision Tree with RFF.}
    \label{fig:\undPrefix_covertype}
  \end{subfigure}%
  \renewcommand{\filePrefix}{\undPrefix/nys}%
  \begin{subfigure}[t]{0.5\linewidth}
    \centering\captionsetup{width=.8\linewidth}\includegraphics[width=\imgscale\linewidth]{Figures/\filePrefix/covertype}
    % \caption{Exp. 4.2 with Covertype. Random Forest outperforms Ensembles of Decision Tree with \Nys.}
    \label{fig:\undPrefix_covertype}
  \end{subfigure}%
  \caption*{Exp. 4.2 with Covertype. Random Forest outperforms Ensembles of
  Decision Tree with RFF (left) and \Nys\ (right).}
\end{figure}


\begin{figure}[H]
  \centering
  \renewcommand{\filePrefix}{\undPrefix/rff}
  \begin{subfigure}[t]{0.5\linewidth}
    \centering\captionsetup{width=.8\linewidth}\includegraphics[width=\imgscale\linewidth]{Figures/\filePrefix/digits}
    % \caption{Exp. 4.2 with Digits. Random Forest outperforms Ensembles of Decision Tree with RFF.}
    \label{fig:\undPrefix_digits}
  \end{subfigure}%
  \renewcommand{\filePrefix}{\undPrefix/nys}%
  \begin{subfigure}[t]{0.5\linewidth}
    \centering\captionsetup{width=.8\linewidth}\includegraphics[width=\imgscale\linewidth]{Figures/\filePrefix/digits}
    % \caption{Exp. 4.2 with Digits. Random Forest outperforms Ensembles of Decision Tree with \Nys.}
    \label{fig:\undPrefix_digits}
  \end{subfigure}
  \caption*{Exp. 4.2 with Digits. Random Forest outperforms Ensembles of Decision Tree with RFF (left) and \Nys\ (right).}
\end{figure}


\begin{figure}[H]
  \centering
  \renewcommand{\filePrefix}{\undPrefix/rff}
  \begin{subfigure}[t]{0.5\linewidth}
    \centering\captionsetup{width=.8\linewidth}\includegraphics[width=\imgscale\linewidth]{Figures/\filePrefix/fall_detection}
    % \caption{Exp. 4.2 with Fall Detection. Random Forest outperforms Ensembles of Decision Tree with RFF.}
    \label{fig:\undPrefix_fall_detection}
  \end{subfigure}%
  \renewcommand{\filePrefix}{\undPrefix/nys}%
  \begin{subfigure}[t]{0.5\linewidth}
    \centering\captionsetup{width=.8\linewidth}\includegraphics[width=\imgscale\linewidth]{Figures/\filePrefix/fall_detection}
    % \caption{Exp. 4.2 with Fall Detection. Random Forest outperforms Ensembles of Decision Tree with \Nys.}
    \label{fig:\undPrefix_fall_detection}
  \end{subfigure}%
  \caption*{Exp. 4.2 with Fall Detection. Random Forest outperforms Ensembles of Decision Tree with RFF (left) and \Nys\ (right).}
\end{figure}


\begin{figure}[H]
  \centering
  \renewcommand{\filePrefix}{\undPrefix/rff}
  \begin{subfigure}[t]{0.5\linewidth}
    \centering\captionsetup{width=.8\linewidth}\includegraphics[width=\imgscale\linewidth]{Figures/\filePrefix/mnist}
    % \caption{Exp. 4.2 with MNIST. Random Forest outperforms Ensembles of Decision Tree with RFF.}
    \label{fig:\undPrefix_mnist}
  \end{subfigure}%
  \renewcommand{\filePrefix}{\undPrefix/nys}%
  \begin{subfigure}[t]{0.5\linewidth}
    \centering\captionsetup{width=.8\linewidth}\includegraphics[width=\imgscale\linewidth]{Figures/\filePrefix/mnist}
    % \caption{Exp. 4.2 with MNIST. Random Forest outperforms Ensembles of Decision Tree with \Nys.}
    \label{fig:\undPrefix_mnist}
  \end{subfigure}
  \caption*{Exp. 4.2 with MNIST. Random Forest outperforms Ensembles of Decision Tree with RFF (left) and \Nys\ (right).}
\end{figure}


\begin{figure}[H]
  \centering
  \renewcommand{\filePrefix}{\undPrefix/rff}
  \begin{subfigure}[t]{0.5\linewidth}
    \centering\captionsetup{width=.8\linewidth}\includegraphics[width=\imgscale\linewidth]{Figures/\filePrefix/pen_digits}
    % \caption{Exp. 4.2 with Pen Digits. Random Forest outperforms Ensembles of Decision Tree with RFF.}
    \label{fig:\undPrefix_pen_digits}
  \end{subfigure}%
  \renewcommand{\filePrefix}{\undPrefix/nys}%
  \begin{subfigure}[t]{0.5\linewidth}
    \centering\captionsetup{width=.8\linewidth}\includegraphics[width=\imgscale\linewidth]{Figures/\filePrefix/pen_digits}
    % \caption{Exp. 4.2 with Pen Digits. Random Forest outperforms Ensembles of Decision Tree with \Nys.}
    \label{fig:\undPrefix_pen_digits}
  \end{subfigure}%
  \caption*{Exp. 4.2 with Pen Digits. Random Forest outperforms Ensembles of Decision Tree with RFF (left) and \Nys\ (right).}
\end{figure}


\begin{figure}[H]
  \centering
  \renewcommand{\filePrefix}{\undPrefix/rff}
  \begin{subfigure}[t]{0.5\linewidth}
    \centering\captionsetup{width=.8\linewidth}\includegraphics[width=\imgscale\linewidth]{Figures/\filePrefix/satellite}
    % \caption{Exp. 4.2 with Satellite. Random Forest outperforms Ensembles of Decision Tree with RFF.}
    \label{fig:\undPrefix_satellite}
  \end{subfigure}%
  \renewcommand{\filePrefix}{\undPrefix/nys}%
  \begin{subfigure}[t]{0.5\linewidth}
    \centering\captionsetup{width=.8\linewidth}\includegraphics[width=\imgscale\linewidth]{Figures/\filePrefix/satellite}
    % \caption{Exp. 4.2 with Satellite. Random Forest outperforms Ensembles of Decision Tree with \Nys.}
    \label{fig:\undPrefix_satellite}
  \end{subfigure}
  \caption*{Exp. 4.2 with Satellite. Random Forest outperforms Ensembles of Decision Tree with RFF (left) and \Nys\ (right).}
\end{figure}


\begin{figure}[H]
  \centering
  \renewcommand{\filePrefix}{\undPrefix/rff}
  \begin{subfigure}[t]{0.5\linewidth}
    \centering\captionsetup{width=.8\linewidth}\includegraphics[width=\imgscale\linewidth]{Figures/\filePrefix/segment}
    % \caption{Exp. 4.2 with Segment. Random Forest outperforms Ensembles of Decision Tree with RFF.}
    \label{fig:\undPrefix_segment}
  \end{subfigure}%
  \renewcommand{\filePrefix}{\undPrefix/nys}%
  \begin{subfigure}[t]{0.5\linewidth}
    \centering\captionsetup{width=.8\linewidth}\includegraphics[width=\imgscale\linewidth]{Figures/\filePrefix/segment}
    % \caption{Exp. 4.2 with Segment. Random Forest outperforms Ensembles of Decision Tree with \Nys.}
    \label{fig:\undPrefix_segment}
  \end{subfigure}%
  \caption*{Exp. 4.2 with Segment. Random Forest outperforms Ensembles of Decision Tree with RFF (left) and \Nys\ (right).}
\end{figure}


\begin{figure}[H]
  \centering
  \renewcommand{\filePrefix}{\undPrefix/rff}
  \begin{subfigure}[t]{0.5\linewidth}
    \centering\captionsetup{width=.8\linewidth}\includegraphics[width=\imgscale\linewidth]{Figures/\filePrefix/vowel}
    % \caption{Exp. 4.2 with Vowel. Random Forest outperforms Ensembles of Decision Tree with RFF.}
    \label{fig:\undPrefix_vowel}
  \end{subfigure}%
  \renewcommand{\filePrefix}{\undPrefix/nys}%
  \begin{subfigure}[t]{0.5\linewidth}
    \centering\captionsetup{width=.8\linewidth}\includegraphics[width=\imgscale\linewidth]{Figures/\filePrefix/vowel}
    % \caption{Exp. 4.2 with Vowel. Random Forest outperforms Ensembles of Decision Tree with \Nys.}
    \label{fig:\undPrefix_vowel}
  \end{subfigure}
  \caption*{Exp. 4.2 with Vowel. Random Forest outperforms Ensembles of Decision Tree with RFF (left) and \Nys\ (right).}
\end{figure}


\begin{figure}[H]
  \centering
  \renewcommand{\filePrefix}{\undPrefix/rff}
  \begin{subfigure}[t]{0.5\linewidth}
    \centering\captionsetup{width=.8\linewidth}\includegraphics[width=\imgscale\linewidth]{Figures/\filePrefix/fashion_mnist}
    % \caption{Exp. 4.2 with Fashion MNIST. Random Forest outperforms Ensembles of Decision Tree with RFF.}
    \label{fig:\undPrefix_vowel}
  \end{subfigure}%
  \renewcommand{\filePrefix}{\undPrefix/nys}%
  \begin{subfigure}[t]{0.5\linewidth}
    \centering\captionsetup{width=.8\linewidth}\includegraphics[width=\imgscale\linewidth]{Figures/\filePrefix/fashion_mnist}
    % \caption{Exp. 4.2 with Fashion MNIST. Random Forest outperforms Ensembles of Decision Tree with \Nys.}
    \label{fig:\undPrefix_vowel}
  \end{subfigure}
  \caption*{Exp. 4.2 with Fashion MNIST. Random Forest outperforms Ensembles of Decision Tree with RFF (left) and \Nys\ (right).}
\end{figure}


%%%%%%%%%%%%%%%%%%%%%%%%%%%%%%%%%%%%%%%%%%%%%%
%%%%%%%%%%%%%%%%%%%%%%%%%%%%%%%%%%%%%%%%%%%%%%
%%%%%%%%%%%%%%%%%%%%%%%%%%%%%%%%%%%%%%%%%%%%%%
%% Termina lo nuevo
%%%%%%%%%%%%%%%%%%%%%%%%%%%%%%%%%%%%%%%%%%%%%%
%%%%%%%%%%%%%%%%%%%%%%%%%%%%%%%%%%%%%%%%%%%%%%
%%%%%%%%%%%%%%%%%%%%%%%%%%%%%%%%%%%%%%%%%%%%%%


% \begin{figure}[H]
%   \centering
%   \begin{subfigure}[t]{0.5\linewidth}
%     \centering\captionsetup{width=.8\linewidth}\includegraphics[width=\imgscale\linewidth]{Figures/\filePrefix/covertype}
%     \caption{Exp. 4.2 with Covertype. Random Forest outperforms Ensembles of Decision Tree with RFF.}
%     \label{fig:\undPrefix_covertype}
%   \end{subfigure}%
%   \begin{subfigure}[t]{0.5\linewidth}
%     \centering\captionsetup{width=.8\linewidth}\includegraphics[width=\imgscale\linewidth]{Figures/\filePrefix/digits}
%     \caption{Exp. 4.2 with Digits. Random Forest outperforms Ensembles of Decision Tree with RFF.}
%     \label{fig:\undPrefix_digits}
%   \end{subfigure}
% \end{figure}
%
%
% \begin{figure}[H]
%   \centering
%   \begin{subfigure}[t]{0.5\linewidth}
%     \centering\captionsetup{width=.8\linewidth}\includegraphics[width=\imgscale\linewidth]{Figures/\filePrefix/fall_detection}
%     \caption{Exp. 4.2 with Fall Detection. Random Forest outperforms Ensembles of Decision Tree with RFF.}
%     \label{fig:\undPrefix_fall_detection}
%   \end{subfigure}%
%   \begin{subfigure}[t]{0.5\linewidth}
%     \centering\captionsetup{width=.8\linewidth}\includegraphics[width=\imgscale\linewidth]{Figures/\filePrefix/mnist}
%     \caption{Exp. 4.2 with MNIST. Random Forest outperforms Ensembles of Decision Tree with RFF.}
%     \label{fig:\undPrefix_mnist}
%   \end{subfigure}
% \end{figure}
%
%
% \begin{figure}[H]
%   \centering
%   \begin{subfigure}[t]{0.5\linewidth}
%     \centering\captionsetup{width=.8\linewidth}\includegraphics[width=\imgscale\linewidth]{Figures/\filePrefix/pen_digits}
%     \caption{Exp. 4.2 with Pen Digits. Random Forest outperforms Ensembles of Decision Tree with RFF.}
%     \label{fig:\undPrefix_pen_digits}
%   \end{subfigure}%
%   \begin{subfigure}[t]{0.5\linewidth}
%     \centering\captionsetup{width=.8\linewidth}\includegraphics[width=\imgscale\linewidth]{Figures/\filePrefix/satellite}
%     \caption{Exp. 4.2 with Satellite. Random Forest outperforms Ensembles of Decision Tree with RFF.}
%     \label{fig:\undPrefix_satellite}
%   \end{subfigure}
% \end{figure}
%
% \begin{figure}[H]
%   \centering
%   \begin{subfigure}[t]{0.5\linewidth}
%     \centering\captionsetup{width=.8\linewidth}\includegraphics[width=\imgscale\linewidth]{Figures/\filePrefix/segment}
%     \caption{Exp. 4.2 with Segment. Random Forest outperforms Ensembles of Decision Tree with RFF.}
%     \label{fig:\undPrefix_segment}
%   \end{subfigure}%
%   \begin{subfigure}[t]{0.5\linewidth}
%     \centering\captionsetup{width=.8\linewidth}\includegraphics[width=\imgscale\linewidth]{Figures/\filePrefix/vowel}
%     \caption{Exp. 4.2 with Vowel. Random Forest outperforms Ensembles of Decision Tree with RFF.}
%     \label{fig:\undPrefix_vowel}
%   \end{subfigure}
% \end{figure}
%
% %%%%%%%%%%%%%%%%%%%%%%%%%%%%%%
%
% \renewcommand{\filePrefix}{\undPrefix/nys}
% \begin{figure}[H]
%   \centering
%   \begin{subfigure}[t]{0.5\linewidth}
%     \centering\captionsetup{width=.8\linewidth}\includegraphics[width=\imgscale\linewidth]{Figures/\filePrefix/covertype}
%     \caption{Exp. 4.2 with Covertype. Random Forest outperforms Ensembles of Decision Tree with \Nys.}
%     \label{fig:\undPrefix_covertype}
%   \end{subfigure}%
%   \begin{subfigure}[t]{0.5\linewidth}
%     \centering\captionsetup{width=.8\linewidth}\includegraphics[width=\imgscale\linewidth]{Figures/\filePrefix/digits}
%     \caption{Exp. 4.2 with Digits. Random Forest outperforms Ensembles of Decision Tree with \Nys.}
%     \label{fig:\undPrefix_digits}
%   \end{subfigure}
% \end{figure}
%
%
% \begin{figure}[H]
%   \centering
%   \begin{subfigure}[t]{0.5\linewidth}
%     \centering\captionsetup{width=.8\linewidth}\includegraphics[width=\imgscale\linewidth]{Figures/\filePrefix/fall_detection}
%     \caption{Exp. 4.2 with Fall Detection. Random Forest outperforms Ensembles of Decision Tree with \Nys.}
%     \label{fig:\undPrefix_fall_detection}
%   \end{subfigure}%
%   \begin{subfigure}[t]{0.5\linewidth}
%     \centering\captionsetup{width=.8\linewidth}\includegraphics[width=\imgscale\linewidth]{Figures/\filePrefix/mnist}
%     \caption{Exp. 4.2 with MNIST. Random Forest outperforms Ensembles of Decision Tree with \Nys.}
%     \label{fig:\undPrefix_mnist}
%   \end{subfigure}
% \end{figure}
%
%
% \begin{figure}[H]
%   \centering
%   \begin{subfigure}[t]{0.5\linewidth}
%     \centering\captionsetup{width=.8\linewidth}\includegraphics[width=\imgscale\linewidth]{Figures/\filePrefix/pen_digits}
%     \caption{Exp. 4.2 with Pen Digits. Random Forest outperforms Ensembles of Decision Tree with \Nys.}
%     \label{fig:\undPrefix_pen_digits}
%   \end{subfigure}%
%   \begin{subfigure}[t]{0.5\linewidth}
%     \centering\captionsetup{width=.8\linewidth}\includegraphics[width=\imgscale\linewidth]{Figures/\filePrefix/satellite}
%     \caption{Exp. 4.2 with Satellite. Random Forest outperforms Ensembles of Decision Tree with \Nys.}
%     \label{fig:\undPrefix_satellite}
%   \end{subfigure}
% \end{figure}
%
% \begin{figure}[H]
%   \centering
%   \begin{subfigure}[t]{0.5\linewidth}
%     \centering\captionsetup{width=.8\linewidth}\includegraphics[width=\imgscale\linewidth]{Figures/\filePrefix/segment}
%     \caption{Exp. 4.2 with Segment. Random Forest outperforms Ensembles of Decision Tree with \Nys.}
%     \label{fig:\undPrefix_segment}
%   \end{subfigure}%
%   \begin{subfigure}[t]{0.5\linewidth}
%     \centering\captionsetup{width=.8\linewidth}\includegraphics[width=\imgscale\linewidth]{Figures/\filePrefix/vowel}
%     \caption{Exp. 4.2 with Vowel. Random Forest outperforms Ensembles of Decision Tree with \Nys.}
%     \label{fig:\undPrefix_vowel}
%   \end{subfigure}
% \end{figure}


\let\major\undefined
\let\minor\undefined

\let\undPrefix\undefined
\let\dotPrefix\undefined
\let\scoPrefix\undefined

\let\filePrefix\undefined


% % Appendix Template

\chapter{Results of experiment 2.5} % Main appendix title

\label{Appendix2-5} % Change X to a consecutive letter; for referencing this appendix elsewhere, use \ref{AppendixX}

\begin{figure}[th]
\centering
\includegraphics[scale=\imgscale]{Figures/2_5/covertype}
\decoRule
\caption[2.5 covertype]{Linear-SVC with RFF and \Nys}
\label{fig:2_5_covertype}
\end{figure}

\begin{figure}[th]
\centering
\includegraphics[scale=\imgscale]{Figures/2_5/digits}
\decoRule
\caption[2.5 digits]{Linear-SVC with RFF and \Nys}
\label{fig:2_5_digits}
\end{figure}

\begin{figure}[th]
\centering
\includegraphics[scale=\imgscale]{Figures/2_5/fall_detection}
\decoRule
\caption[2.5 fall\tu detection]{Linear-SVC with RFF and \Nys}
\label{fig:2_5_fall_detection}
\end{figure}

\begin{figure}[th]
\centering
\includegraphics[scale=\imgscale]{Figures/2_5/mnist}
\decoRule
\caption[2.5 mnist]{Linear-SVC with RFF and \Nys}
\label{fig:2_5_mnist}
\end{figure}

\begin{figure}[th]
\centering
\includegraphics[scale=\imgscale]{Figures/2_5/pen_digits}
\decoRule
\caption[2.5 pen\tu digits]{Linear-SVC with RFF and \Nys}
\label{fig:2_5_pen_digits}
\end{figure}

\begin{figure}[th]
\centering
\includegraphics[scale=\imgscale]{Figures/2_5/satellite}
\decoRule
\caption[2.5 satellite]{Linear-SVC with RFF and \Nys}
\label{fig:2_5_satellite}
\end{figure}

\begin{figure}[th]
\centering
\includegraphics[scale=\imgscale]{Figures/2_5/segment}
\decoRule
\caption[2.5 segment]{Linear-SVC with RFF and \Nys}
\label{fig:2_5_segment}
\end{figure}

\begin{figure}[th]
\centering
\includegraphics[scale=\imgscale]{Figures/2_5/vowel}
\decoRule
\caption[2.5 vowel]{Linear-SVC with RFF and \Nys}
\label{fig:vowel}
\end{figure}

% % Appendix Template

\chapter{Results of experiment 2.6} % Main appendix title

\label{Appendix2-6} % Change X to a consecutive letter; for referencing this appendix elsewhere, use \ref{AppendixX}

\begin{figure}[ht]
  \centering
  \begin{subfigure}[b]{0.5\linewidth}
    \centering\captionsetup{width=.8\linewidth}\includegraphics[width=\imgscale\linewidth]{Figures/2_6/covertype}
    \caption{prueba covertype}
    \label{fig:2_6_covertype}
  \end{subfigure}%
  \begin{subfigure}[b]{0.5\linewidth}
    \centering\captionsetup{width=.8\linewidth}\includegraphics[width=\imgscale\linewidth]{Figures/2_6/digits}
    \caption{prueba digits}
    \label{fig:2_6_digits}
  \end{subfigure}
\end{figure}


\begin{figure}[ht]
  \centering
  \begin{subfigure}[b]{0.5\linewidth}
    \centering\captionsetup{width=.8\linewidth}\includegraphics[width=\imgscale\linewidth]{Figures/2_6/fall_detection}
    \caption{prueba fall-detection}
    \label{fig:2_6_fall_detection}
  \end{subfigure}%
  \begin{subfigure}[b]{0.5\linewidth}
    \centering\captionsetup{width=.8\linewidth}\includegraphics[width=\imgscale\linewidth]{Figures/2_6/mnist}
    \caption{prueba mnist}
    \label{fig:2_6_mnist}
  \end{subfigure}
\end{figure}


\begin{figure}[ht]
  \centering
  \begin{subfigure}[b]{0.5\linewidth}
    \centering\captionsetup{width=.8\linewidth}\includegraphics[width=\imgscale\linewidth]{Figures/2_6/pen_digits}
    \caption{prueba pen-digits}
    \label{fig:2_6_pen_digits}
  \end{subfigure}%
  \begin{subfigure}[b]{0.5\linewidth}
    \centering\captionsetup{width=.8\linewidth}\includegraphics[width=\imgscale\linewidth]{Figures/2_6/satellite}
    \caption{prueba satellite}
    \label{fig:2_6_satellite}
  \end{subfigure}
\end{figure}

\begin{figure}[ht]
  \centering
  \begin{subfigure}[b]{0.5\linewidth}
    \centering\captionsetup{width=.8\linewidth}\includegraphics[width=\imgscale\linewidth]{Figures/2_6/segment}
    \caption{prueba segment}
    \label{fig:2_6_segment}
  \end{subfigure}%
  \begin{subfigure}[b]{0.5\linewidth}
    \centering\captionsetup{width=.8\linewidth}\includegraphics[width=\imgscale\linewidth]{Figures/2_6/vowel}
    \caption{prueba vowel}
    \label{fig:2_6_vowel}
  \end{subfigure}
\end{figure}

%
% % Appendix Template

\chapter{Results of experiment 2.7} % Main appendix title

\label{Appendix2-7} % Change X to a consecutive letter; for referencing this appendix elsewhere, use \ref{AppendixX}

\begin{figure}[ht]
  \centering
  \begin{subfigure}[b]{0.5\linewidth}
    \centering\includegraphics[width=\imgscale\linewidth]{Figures/2_7/covertype}
    \caption{prueba covertype}
    \label{fig:2_7_covertype}
  \end{subfigure}%
  \begin{subfigure}[b]{0.5\linewidth}
    \centering\includegraphics[width=\imgscale\linewidth]{Figures/2_7/digits}
    \caption{prueba digits}
    \label{fig:2_7_digits}
  \end{subfigure}
\end{figure}


\begin{figure}[ht]
  \centering
  \begin{subfigure}[b]{0.5\linewidth}
    \centering\includegraphics[width=\imgscale\linewidth]{Figures/2_7/fall_detection}
    \caption{prueba fall-detection}
    \label{fig:2_7_fall_detection}
  \end{subfigure}%
  \begin{subfigure}[b]{0.5\linewidth}
    \centering\includegraphics[width=\imgscale\linewidth]{Figures/2_7/mnist}
    \caption{prueba mnist}
    \label{fig:2_7_mnist}
  \end{subfigure}
\end{figure}


\begin{figure}[ht]
  \centering
  \begin{subfigure}[b]{0.5\linewidth}
    \centering\includegraphics[width=\imgscale\linewidth]{Figures/2_7/pen_digits}
    \caption{prueba pen-digits}
    \label{fig:2_7_pen_digits}
  \end{subfigure}%
  \begin{subfigure}[b]{0.5\linewidth}
    \centering\includegraphics[width=\imgscale\linewidth]{Figures/2_7/satellite}
    \caption{prueba satellite}
    \label{fig:2_7_satellite}
  \end{subfigure}
\end{figure}

\begin{figure}[ht]
  \centering
  \begin{subfigure}[b]{0.5\linewidth}
    \centering\includegraphics[width=\imgscale\linewidth]{Figures/2_7/segment}
    \caption{prueba segment}
    \label{fig:2_7_segment}
  \end{subfigure}%
  \begin{subfigure}[b]{0.5\linewidth}
    \centering\includegraphics[width=\imgscale\linewidth]{Figures/2_7/vowel}
    \caption{prueba vowel}
    \label{fig:2_7_vowel}
  \end{subfigure}
\end{figure}

% % Appendix Template

\chapter{Results of experiment 2.2} % Main appendix title

\label{Appendix2-2} % Change X to a consecutive letter; for referencing this appendix elsewhere, use \ref{AppendixX}

\begin{figure}[th]
\centering
\includegraphics[scale=\imgscale]{Figures/2_2/covertype}
\decoRule
\caption[2.2 covertype]{Logistic Regression with Black Bag model}
\label{fig:2_2_covertype}
\end{figure}

\begin{figure}[th]
\centering
\includegraphics[scale=\imgscale]{Figures/2_2/digits}
\decoRule
\caption[2.2 digits]{Logistic Regression with Black Bag model}
\label{fig:2_2_digits}
\end{figure}

\begin{figure}[th]
\centering
\includegraphics[scale=\imgscale]{Figures/2_2/fall_detection}
\decoRule
\caption[2.2 fall\tu detection]{Logistic Regression with Black Bag model}
\label{fig:2_2_fall_detection}
\end{figure}

\begin{figure}[th]
\centering
\includegraphics[scale=\imgscale]{Figures/2_2/mnist}
\decoRule
\caption[2.2 mnist]{Logistic Regression with Black Bag model}
\label{fig:2_2_mnist}
\end{figure}

\begin{figure}[th]
\centering
\includegraphics[scale=\imgscale]{Figures/2_2/pen_digits}
\decoRule
\caption[2.2 pen\tu digits]{Logistic Regression with Black Bag model}
\label{fig:2_2_pen_digits}
\end{figure}

\begin{figure}[th]
\centering
\includegraphics[scale=\imgscale]{Figures/2_2/satellite}
\decoRule
\caption[2.2 satellite]{Logistic Regression with Black Bag model}
\label{fig:2_2_satellite}
\end{figure}

\begin{figure}[th]
\centering
\includegraphics[scale=\imgscale]{Figures/2_2/segment}
\decoRule
\caption[2.2 segment]{Logistic Regression with Black Bag model}
\label{fig:2_2_segment}
\end{figure}

\begin{figure}[th]
\centering
\includegraphics[scale=\imgscale]{Figures/2_2/vowel}
\decoRule
\caption[2.2 vowel]{Logistic Regression with Black Bag model}
\label{fig:vowel}
\end{figure}


%----------------------------------------------------------------------------------------
%	BIBLIOGRAPHY
%----------------------------------------------------------------------------------------

% \printbibliography[heading=bibintoc]

% \printbibliography[nottype=online,heading=subbibliography,title={Articles}]
% \printbibliography[type=online,heading=subbibliography,title={Datasets}]

\printbibliography[nottype=online,heading=bibintoc,title={References}]
\printbibliography[type=online,heading=bibintoc,title={Datasets}]

%----------------------------------------------------------------------------------------

\end{document}
