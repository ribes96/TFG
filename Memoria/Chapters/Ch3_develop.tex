% Chapter Template

\chapter{Project Development} % Main chapter title

\label{Chapter3} % Change X to a consecutive number; for referencing this chapter elsewhere, use \ref{ChapterX}

%----------------------------------------------------------------------------------------
%	SECTION 1
%----------------------------------------------------------------------------------------

\section{General Idea}
\begin{note}
  \begin{itemize}
    \item Hemos visto que se puede sacar una aproximación aleatoria de la
    función implícita de un shift invariant kernel. Esto tiene 2 ventajas
    \begin{itemize}
      \item Podemos transformar los datos directamente
      \item Podemos producir pequeñas variaciones de un mismo dataset, todas
      ellas válidas
    \end{itemize}
    \item Las 4 tipos de modelos que he definido. Referencia a la foto
    \item ¿Por qué he cogido estos 4 modelos? ¿No podrían haber sido otros?
    ¿Que tienen estos de bueno? Me he inspirado en Random Forest
    \item Hay por ahí algún paper que compara RFF y \Nys
  \end{itemize}
\end{note}
\begin{note}
  \subsection{State of the art con las RFF}
\end{note}
\begin{note}
  \begin{itemize}
    \item Se ha trabajado poco con ellas. Solo he encontrado 2 usos:
    \begin{itemize}
      \item Stacked kernel network (referencia): usarlas junto a una red
      neuronal para tener más niveles de aprendizaje no lineal
      \item RFF with SVM (referencia): usar una SVM sin kernel con los datos
      mapeados usando RFF
    \end{itemize}
  \end{itemize}
\end{note}
\begin{note}
  \subsection{State of the art con las \Nys}
\end{note}


\section{Hyper-parameters}
\begin{note}
  \begin{itemize}
    \item Existen los siguientes:
    \begin{itemize}
      \item min-impurity-decrease para DT
      \item C para SVM
      \item gamma para RFF y \Nys
      \item cantidad de features para RFF y \Nys
      \item cantidad de estimadores para ensembles
    \end{itemize}
    \item Hemos usado los siguientes valores:
    \begin{itemize}
      \item Cantidad de features a 500
      \item Cantidad de estimadores a 50
      \item En modelos simples, el parámetro por crossvalidation
      \item En modelos simples con RFF, el parámetro por crossvalidation
      y una gamma que sobreajuste
      \item En modelos con ensemble, parámetros que sobreajusten y la gamma
      por crossvalidation
      \item En RBF-SVM, la gamma por gamest y el parámetro por crossvalidation
    \end{itemize}
  \end{itemize}
\end{note}
\section{Hypothesis}
\begin{note}
  \begin{enumerate}
    \item Podemos aproximar bien una RBF-SVM
    \item Puede tener sentido hacer ensembles con otros modelos a DT
    \item RFF + Bootstrap puede ser malo
    \item Si el modelo no se basa en productos escalares no se
    feneficiará tanto
  \end{enumerate}
\end{note}
\begin{note}
  \subsection{Planteamiento de los experimentos}
\end{note}
\begin{note}
  \begin{enumerate}
    \item Hipótesis: Aproximar RBF-SVM
    \begin{enumerate}
      \item Comparar una RBF-SVM con SVM normal que use RFF
    \end{enumerate}
    \item Hipótesis: Ensembles con otros
    \begin{enumerate}
      \item Logit normal vs. Logit con RFF
      \item Logit normal vs. Logit con RFF Black Bag
      \item Logit normal vs. Logit con RFF Grey Bag
      \item Logit normal vs. Logit con RFF Grey Ensemble
      \hrule
      \item Linear-SVM vs Linear-SVM con RFF
      \item Linear-SVM vs Linear-SVM con RFF Black Bag
      \item Linear-SVM vs Linear-SVM con RFF Grey Bag
      \item Linear-SVM vs Linear-SVM con RFF Grey Ensemble
    \end{enumerate}
    \item Hipótesis: RFF + Bootstrap
    \begin{enumerate}
      \item Logit con RFF Grey Bag vs Logit con RFF Grey Ensemble
      \item Logit con RFF Black Bag vs Logit con RFF Black Ensemble (los
      dos con un solo estimador)
      \hrule
      \item Linear-SVM con RFF Grey Bag vs Linear-SVM con RFF Grey Ensemble
      \item Linear-SVM con RFF Black Bag vs Linear-SVM con RFF Black Ensemble (los
    \end{enumerate}
    \item Hipótesis: DT + RFF
    \begin{itemize}
      \item DT vs DT con RFF
      \item DT vs DT con RFF Black Bag
      \item DT vs DT con RFF Black Ensemble
      \item DT vs DT con RFF Grey Bag
      \item DT vs DT con RFF Grey Ensemble
    \end{itemize}
  \end{enumerate}
\end{note}

\section{Datasets}
\begin{note}
  \begin{itemize}
    \item 8 Datasets
    \item Normalizados
    \item Únicamente tienen variables numéricas, no categóricas
    \item Únicamente problemas de clasificación
    \item Algunas cosas particulares que he hecho:
    \begin{itemize}
      \item Mezclar datos de train y de test para luego hacer mi propia
      separación
      \item Cuando había poca presencia de una clase, hacer un resampling para
      igualar las cantidades
      \item No trabajar cosas como el skiwness o los outliers
      \item Eliminar columnas en las que todo eran 0
      \item Reducir el conjunto de instancias
    \end{itemize}
  \end{itemize}
\end{note}
