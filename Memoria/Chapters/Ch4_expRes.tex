% Chapter Template

\chapter{Experimental Results} % Main chapter title

\label{Chapter4} % Change X to a consecutive number; for referencing this chapter elsewhere, use \ref{ChapterX}

%----------------------------------------------------------------------------------------
%	SECTION 1
%----------------------------------------------------------------------------------------

\begin{pre-delivery}
  hi
\end{pre-delivery}

\begin{note}
  \section{Enfrentar resultados 2 a 2}
\end{note}
\begin{note}
  \section*{Exp 1-1}
  \begin{itemize}
    \item Nunca jamás conseguimos sacar mejores resultados que una RBF-SVM
    \item Pero los resultados que conseguimos son bastante parecidos en algunos
    casos
    \item En el experimento que hamos hecho, los tiempos de los métodos lineales
    son mucho mayores que los de la RBF-SVM
    \item Pero hay que atribuirlo al pequeño tamaño del dataset que hemos usado.
    En un dataset más grande sí se notaría
    \item Efectivamente, he realizado el experimento en fashion-mnist, y no hay
    comparación
  \end{itemize}

  \section*{Exp 2-1}
  \begin{itemize}
    \item En general sí conseguimos mejorar sustancialmente un logit
    \item Pero los tiempos son un poco mayores
    \item Con datasets mayores, es de esperar que los tiempos no se disparen
    demasiado
  \end{itemize}

\section*{Exp 2-2}
\begin{itemize}
  \item En los mismos datasets en los que un solo logit no había conseguido hacer
  nada, aquí tampoco han hecho nada
  \item Entre logit normal y logit con ensemble, la precisión es más o menos la
  misma, pero el tiempo es mucho mayor
  \item En el caso logit, realmente puede deberse a que no hemos hecho nada
  para sobreajustar más.
\end{itemize}

\section*{2-3}
\begin{itemize}
  \item Exactamente igual que el anterior
\end{itemize}

\section*{2-4}
\begin{itemize}
  \item Exactamente igual que el anterior
\end{itemize}

\section*{2-5}
\begin{itemize}
  \item Conseguimos mejorar sustancialmente la precisión de la mayoría de
  datasets
  \item Los tiempos son mayores, pero todavía son aceptables
  \item Suponemos que en datasets más grandes los tiempo serían más parecidos
\end{itemize}

\section*{2-6}
\begin{itemize}
  \item El dataset que se resistía se sigue resistiendo cuando hacemos un
  ensemble
  \item Ahora los tiempo sí que són muchísimo más grandes, quizá no sale a cuenta
  hacer ensemble
  \item Para ver si sale a cuenta hacer ensemble, habrá que verlo más adelante
\end{itemize}

\section*{2-7}
\begin{itemize}
  \item Exactamente igual que el anterior
\end{itemize}

\section{2-8}
\begin{itemize}
  \item Exactamente igual que el anterior
\end{itemize}

\section*{3-1}
\begin{itemize}
  \item No podemos decir que haya una ensemble que claramente sea mejor que los
  demás
  \item Da la impresión que en algunos casos el grey es mejor que los otros,
  pero no es nada concluyente
\end{itemize}

\section*{3-2}
\begin{itemize}
  \item Da la impresión que los black ensembles son un poco mejor que los grey,
  pero no es nada concluyente
\end{itemize}

\section*{3-3}
\begin{itemize}
  \item Los resultados no son nada concluyentes
\end{itemize}

\section*{3-4}
\begin{itemize}
  \item Exactamente igual que el anterior
\end{itemize}

\section*{4-1}
\begin{itemize}
  \item Usar rff con un solo árbol no es nada beneficioso
  \item Tanto los errores como los tiempos son más altos
\end{itemize}

\section*{4-2}
\begin{itemize}
  \item En algunos casos ha mejorado sustancialmente, y en otros no
  \item Pero estamos comparando un solo árbol con 50: claramente los 50
  tendrían que ser siempre mejores, y ese no es el caso
  \item Los tiempos no tienen ninguna comparación
  \item En algunos casos, incluso empeora el usar un ensemble
\end{itemize}

\section*{4-3}
\begin{itemize}
  \item Exactamente igual que el anterior
\end{itemize}

\section*{4-4}
\begin{itemize}
  \item Exactamente igual que el anterior
\end{itemize}

\section*{4-5}
\begin{itemize}
  \item Exactamente igual que el anterior
\end{itemize}

\end{note}
\begin{note}
  \section{Contrastar hipótesis con resultados}
\end{note}
