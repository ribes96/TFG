% Chapter Template

\chapter{Introduction} % Main chapter title

\label{Chapter1} % Change X to a consecutive number; for referencing this chapter elsewhere, use \ref{ChapterX}

% \usepackage{mystyle}

%----------------------------------------------------------------------------------------
%	SECTION 1
%----------------------------------------------------------------------------------------

\section{Problem to solve}

\begin{note}
  \begin{itemize}
    \item Trade-off between accuracy and train time is not good
  \end{itemize}
\end{note}



%----------------------------------------------------------------------------------------
%	SECTION 2
%----------------------------------------------------------------------------------------

\section{Why is it important?}

\begin{note}
  \begin{itemize}
    \item Avances en este campo permitirían usarlo en otras ciencias como medicina,
    economía, sociedad
    \item Muchas tareas que ahora tiene que hacer un humano podría hacerlas una
    máquina, ahorrando tiempo y dinero
  \end{itemize}
\end{note}


%----------------------------------------------------------------------------------------
%	SECTION 3
%----------------------------------------------------------------------------------------

\section{Project proposal}

\begin{note}
  \begin{itemize}
    \item Existe una batería de técnicas que son buenas, pero que nadie las
    ha combinado. Son:
    \begin{itemize}
      \item Modelos simples
      \item Ensembles
      \item kernel trick
      \item Aproximaciones de kernel
    \end{itemize}
    \item La propuesta es combinar todo esto para mejorar el trade-off
    \item Sostenemos las siguientes hipótesis:
    \begin{itemize}
      \item Se podría hacer un ensemble con modelos distintos a DT
      \item Se puede aproximar una RBF-SVM pero con el coste de una lineal
      \item RFF + Bootstrap quizá es demasiado aleatorio
      \item Los modelos que no se basan en productos escalares no se
      beneficiarán tanto de usar RFF
    \end{itemize}
    \item Lo que se hará en cada capítulo del trabajo
  \end{itemize}
\end{note}
