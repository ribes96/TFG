% Chapter Template

\chapter{Conclusion and Future Directions} % Main chapter title

\label{Chapter5} % Change X to a consecutive number; for referencing this chapter elsewhere, use \ref{ChapterX}

%----------------------------------------------------------------------------------------
%	SECTION 1
%----------------------------------------------------------------------------------------

\begin{note}
  \begin{itemize}
    \item Problemas de regresión
    \item Aproximar otros kernels a RBF
    \item Ver el comportamiento con problemas que no sean tan bonitos (con
    missings, clases desbalanceadas, etc)
    \item Otros tipos de ensembles, como el boosting
  \end{itemize}
\end{note}

\begin{note}
  \begin{itemize}
  \item Todo lo que hemos hecho solo sirve con datasets muy grandes. Para
  pequeños, en general salimos perdiendo
  \item Hemos conseuido hacerle un boost a logit
  \item Si usamos RFF, en general no sale nada a cuenta hacer un
  ensemble. No se mejora demasiado
  \item Sí que podemos aproximar una RBF-SVM con una lineal a coste lineal
\end{itemize}

En general los únicos éxitos de este trabajo son:
\begin{itemize}
  \item Ahora podemos hacer un ensemble de logit y de svm, que antes no se podría
  \item también hemos conseguido mejorar un poco un solo logit y svm
  \item Hemos aprendido que da igual el tipo de ensemble que cojamos
\end{itemize}
\end{note}
