% Chapter Template

\chapter{Background Information and Theory} % Main chapter title

\label{Chapter2} % Change X to a consecutive number; for referencing this chapter elsewhere, use \ref{ChapterX}

%----------------------------------------------------------------------------------------
%	SECTION 1
%----------------------------------------------------------------------------------------

\section{Machine Learning}
\begin{note}
  \begin{itemize}
    \item Una definición rápida
    \item Clasificación y regresión
    \item Cross-validation
    \item Qué son los datos de train y test, y por qué se hace esa partición
    \item Qué es el sobre-ajuste
  \end{itemize}
\end{note}

\begin{pre-delivery}
  Machine Leearning uses statistical and mathematical models to give answers to
  problems when there is no known formula of procedure to compute the answer.
  In the subfield of Supervised Learning, the objective is to predict a numerical
  or categorical variable in response to some input data, and the way of doing
  it is to feed the model with lots of different examples for which we already
  know the correct answer, and we expect the models to be able to predict
  the correct answer to instances that it hasn't seen before. When it does,
  we say that the model is able to generalize.

  When a model is trained with some data, there is always a risk of overfiting.
  For a model to overfit means that it adjusts very well to the data that is
  has seen, but can't predict the correct answer to new, unseen data. This
  happens because it has not only from the relevant information, but also
  from the random noise that the data had, and so it can only memorise, but
  not generalize.

  For this reason, when a Machine Learning algorithm is trained the data
  is split in two subsets, a \textit{Training  dataset} and a \textit{Testing
  dataset}. The training dataset will be used to train the model, while the
  Testing datasets will be used only to check it. If a model has generalized
  well, it will achieve a good accuracy score on both the training and the
  testing dataset, but if it has overfitted it will show good results in the
  training dataset and bad ones in the testing dataset.

  Many models need some parameters to tune the behaviour of the algorithm. For
  example, some of them are used to adjust how much a model will fit to the data.
  We usually call these ``hyperparameters''. The correct value for them is not
  straightforward, and it is normally chosen with a process called
  ``cross-validation''. This process consists of splitting the training dataset
  in many subsets and check many possible values for the hyperparameters in order
  to see which one gets a higher accuracy with unseen data.
\end{pre-delivery}

\section{Some currently used Machine Learning models}


% \begin{note}
  % \section{Review de los principales modelos que existen}
% \end{note}
  \subsection{Decision Tree}
  \begin{note}
    \begin{itemize}
      \item No se basa en productos escalares
      \item Es extremadamente rápido
      \item Es más fácil de interpretar que otros modelos
      \item Es extremadamente inestable
      \item Cuando se hace un Random Forest, se randommiza un poco, de modo que
      árboles distintos entrenados con los mismos datos pueden ser destintos
      \item Es un modelo no lineal
    \end{itemize}
  \end{note}
  \begin{pre-delivery}
    Decision Tree is a predictive model which uses the training data to build
    a tree where each node splits the data in two sets accoding to some
    feature, and the leafs contain the set of instances that belong to some class
    (in classification problems) or that has a similar numerical response variable
    (for regression problems).

    To predict the answer to a new instance, it uses the features to ``decide''
    the nodes to cross until it reaches a leaf. The response given is the
    most prevailing class in the leaf for classification problems, or the mean
    of the values of the rest of the instances in the leaf.

    Decision Trees have the advantages that it is easy to interpret the
    tree produced and that it is very fast to build the tree. The way to avoid
    overfitting is to limit its growth.

    These models are very unstable. This means that small differences in the
    training data can produce very different Decision Trees. This property
    is very useful to build an ensemble of estimators to produce better aswers.
    Random Forest is an algorithm that trains many Decision Trees with some sort
    of randomization.
  \end{pre-delivery}
  \subsection{Logistic Regression}
  % \begin{delivery}
  %   Hola que tal
  % \end{delivery}
  \subsection{Support Vector Machines}
  \begin{note}
    \begin{itemize}
      \item Inicialmente pensadas para clasificación en 2 clases
      \item Pero se puede más clases con \eng{one-vs-rest} y también hay
      formas de hacer regresión
      \item Se basa únicamente en el producto escalar de sus entradas
      \item Intenta separar los datos con un híper-plano
      \item Actualmente es poco eficiente usarlas porque su coste s cúbico
      con la cantidad de entradas.
      \item Las fórmulas que quiere optimizar
    \end{itemize}
  \end{note}

  \begin{pre-delivery}
    Support Vector Machine (SVM) is a model that finds in hyperplane that
    divides the data in two sets. In two-class classification problems, each
    side of the hyperplane contains the instances of each of the classes.
    It does so by converting the problem to an optimization one.

    Given some data
    $D = \{\bm{\chi}, \bm{y}\}$
    , where
    $\bm{\chi} = \{\bm{x}_1, \ldots \bm{x}_n\}$, $\bm{x}_i \in \reals^d$, $\bm{y} = \{-1, +1\}^n$
    , the optimization problem consists on finding $\bm{\alpha} \in \reals^n$
    the maximises

\begin{equation}
  L = \sum_{i = 1}^n\alpha_i -\frac{1}{2}\sum_{i = 1}^n\sum_{j = 1}^n\alpha_i\alpha_jy_iy_j\vx_i^\transp\vx_j
\end{equation}

subject to

\begin{align}
  0 \leq \alpha_i \leq C; \forall i\\
  \sum_{i = 1}^n \alpha_iy_i = 0
\end{align}

$C$ is an hyperparameter to tune the ammount of penalization for miss-classified
instances.

If we compute
\begin{equation}
  w = \sum_{i = 1}^n\alpha_iy_i\vx_i
\end{equation}
and

\begin{equation}
  b = y_i - w\vx_i
\end{equation}
for any $i$ so that $\alpha_i \neq 0$, we can compute the class of $\vx_0$ with
$sign(w\vx_0 + b)$.

Note that this algorithm just uses the dot product of the input data, not the
data itself. This property allows us to use the Kernel Trick with them.
See \ref{sec:kern-trick}

\end{pre-delivery}


\section{Ensemble Methods}
  % \subsection{Bagging}
    \begin{note}
      \begin{itemize}
        \item Bagging
        \begin{itemize}
          \item Inventado por Leo Breiman (referencia)
          \item Pretende reducir el sesgo
          \item Wikipedia dice que pretende reducir la varianza
          \item Es el boosting el que pretende reducir el sesgo
          \item Entrenamiento de los estimadores es independiente, se podría
          hacer en paralelo
          \item Actualmente casi solo se usa con DT, debido a su inestabilidad
        \end{itemize}
        \item Bootstrap
        \begin{itemize}
          \item Intenta solucionar el problema de que para bagging es bueno
          que los estimadores sean distintos
          \item Idealmente usaríamos un dataset distinto para cada estimador
          \item Consiste en hacer un resalmpling con repetición
          \item Si la cantidad de instancias del original es la misma que la de cada uno
          de los subconjuntos, se espera que la proporción de elementos úncos sea de
          $1 - \frac{1}{e} \approx 0.632$.
          \item Si el conjunto original tiene $n$ elementos, y tu haces un subconjunto
          de tamaño $r$, puedes esperar que la proporción de elementos del original que
          sí tienen presencia en el nuevo sea de $1 - e^{-\frac{r}{n}}$
        \end{itemize}
        \item Random Forest
      \end{itemize}
    \end{note}

  \begin{pre-delivery}
    Ensemble methods are a technique used in Machine Learning to reduce the
    overall accuracy error of a basic classification or regression model. The
    idea is that a comitee of models is expected to learn better than a single
    one.

    Some ensemble methods are focused on decreasing the error caused by the
    variance of the model. One example is \textit{Bagging}. Others are focused
    on error decreasing the bias error, like \textit{Boosting}.

    In Bagging, every model in the ensemble vote with equal weight. Thus, it is
    important to promote the variance among each of the models, since not doing
    it would be equivalent to training just one model. Ideally, one would train
    each of the models with totally different datasets, with no correlation
    among them. But in practice this is not always possible, because of a
    limited number of instances to train. One alternative is to use a
    technique called \textit{Bootstrap}. Bootstrap allows to generate
    many different instances of a dataset by performing a resampling.

    Given a dataset $D$ of size $n$, Bootstrap generates $m$ new datasets
    $D_i$ of size $n$ by sampling instances from $D$ uniformly and with
    replacement. This means that some of the instances in $D$ may be repeated
    in $D_i$, and others may not appear at all. With a large $n$, it is expected
    that each dataset $D_i$ will contain $63.2 \% $ of the instances in $D$.

    Theoretically Bagging could be used with any kind of method. However, for
    most of them Boostrap is not enought to decorrelate the estimators.
    In practice, Bagging is mostly used with Decision Tree, given that this
    method produces very different trees with a small variation in the data.
    Random Forest is an algorithm that trains many Decision Trees with a
    Bagging. Instead of building the tree in a deterministic way, in each
    split it choses a random subset of features on which to perform the
    separation. Besides, it lets the estimators overfit, since it has a positive
    impact in reducing the overall variance of the Forest.
  \end{pre-delivery}

\section{The kernel trick}
\label{sec:kern-trick}




\begin{note}
  \begin{itemize}
    \item Teorema de Bochner
    \item El kernel RBF
    \begin{itemize}
      \item Su fórmula es \ldots
      \item Equivalencia entre $\gamma$ y $\sigma$
      \item La noción de similitud que tiene
      \item \Hspace\ es de dimensionalidad infinita
      \item Permite ajustarse infinitamente a los datos, tuneando el
      híper-parámetro
      \item $\sigma$ más pequeño, más sobreajuste
      \item $\gamma$ más grande, más sobreajuste
    \end{itemize}
  \end{itemize}
\end{note}

\begin{pre-delivery}
  A Kernel is a function that equals to the inner product of inputs mapped into
  some Hilbert Space
  \footnote{A Hilbert space is just a generalization of the Euclidean Space which contains
  the structure of an inner product that allows length and angle to be
  measured.}
  , i.e:
  \begin{equation}
  \kernel(x,y) = \langle \phi(x)\phi(y) \rangle
\end{equation}
% A Hilbert space is just a generalization of the Euclidean Space which contains
% the structure of an inner product that allows length and angle to be measured.

They are interesting in Machine Learning because we don't need to know the
explicit function $\phi(\cdot)$. In fact, $\phi(\cdot)$ could map the data to
a Hilbert Space with infinite dimensions, and we could still compute
$\phi(\vx)\phi(\vy)$ through the kernel $\kernel$

Support Vector Machines can benefit a lot of Kernel Functions. SVMs solve an
optimization problem to maximise

\begin{equation}
  L = \sum_{i = 1}^n\alpha_i -\frac{1}{2}\sum_{i = 1}^n\sum_{j = 1}^n\alpha_i\alpha_jy_iy_j\vx_i^\transp\vx_j
\end{equation}

in order to find an hyperplane that separates the data points in two classes.
But with some problems there may not exist such hyperplane, and so it would
be needed to map the data to a different feature space. If we did that, then
the function to maximise would be

\begin{equation}
  L = \sum_{i = 1}^n\alpha_i -\frac{1}{2}\sum_{i = 1}^n\sum_{j = 1}^n\alpha_i\alpha_jy_iy_j\phi(\vx_i)^\transp\phi(\vx_j)
\end{equation}

As we said previously, SVMs don't work with the data points alone, but just with
their inner products. Thus, a Kernel could be used to define the optimization
problem as

\begin{equation}
  L = \sum_{i = 1}^n\alpha_i -\frac{1}{2}\sum_{i = 1}^n\sum_{j = 1}^n\alpha_i\alpha_jy_iy_j\kernel(\vx_i, \vx_j)
\end{equation}

This approach has one big advantage:
% first, we don't need to explicitly
% compute $\phi(\vx)^\transp\phi(\vy)$, which could have a high cost if the
% new dimensionality was too big.
as long as the learning technique relies
only on the inner product of the input, the underlying mapping $\phi(\cdot)$
does not need to be explicitly calculated and can, in fact, be unknown.

Kernel functions can be characterised with the Mercer's condition
\cite{mercer1909functions}. It says that given a function $\kernel(\vx, \vy)$,
there exists a mapping $\phi(\cdot)$ so that
$\kernel(\vx, \vy) = \phi(\vx)\phi(\vy)$
if and only if for any $g(\vx)$ such that $\int g(\vx)^2 d\vx$ is finite then
$\kernel(\vx, \vy)g(\vx)g(\vy) \geq 0$.

There are many known Kernels. One that is very popular is the Radial Basis
Function Kernel, RBF. This kernel is defined as:
\begin{equation}
\kernel(\vx,\vy) = \semiRbf
\end{equation}
where $\gamma$ is a free parameter. The value of this Kernel decreases with the
euclidean distance of the parameters, so it can be interpreted as a measure
of similarity. The feature space of this kernel has infinite number of
dimensions.

When a kernel is used with an SVM, the answer can be computed with
$sign\left(\sum_{i = 1}^n \alpha_iy_i\kernel(\vx_i, \vx)\right)$.
SVMs using the RBF kernel have a huge ability to fit to the data, and is able
to separate classes for very difficult problems. The problem is that the
optimization of the function 

\end{pre-delivery}
  % \subsection{The RBF kernel}

\section{Random Fourier Features}
\section{\Nys}
