% Chapter Template

\chapter{Background Information and Theory} % Main chapter title

\label{Chapter2} % Change X to a consecutive number; for referencing this chapter elsewhere, use \ref{ChapterX}

%----------------------------------------------------------------------------------------
%	SECTION 1
%----------------------------------------------------------------------------------------

\section{Machine Learning}
\begin{note}
  \begin{itemize}
    \item Clasificación y regresión
    \item Cross-validation
    \item Qué son los datos de train y test, y por qué se hace esa partición
    \item Qué es el sobre-ajuste
  \end{itemize}
\end{note}

\begin{note}
  \section{Review de los principales modelos que existen}
\end{note}
  \subsection{Decision Tree}
  \subsection{Logistic Regression}
  \subsection{Support Vector Machines}

\section{Ensemble Methods}
  \subsection{Bagging}
    \begin{note}
      \begin{itemize}
        \item Bootstrap
        \item Random Forest
      \end{itemize}
    \end{note}

\section{The kernel trick}
  \subsection{The RBF kernel}

\section{Random Fourier Features}
\section{Nystroem}
