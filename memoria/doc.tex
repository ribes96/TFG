\documentclass{article}
\usepackage[utf8]{inputenc}
% \usepackage[spanish]{babel}
\usepackage[english]{babel}

\usepackage{biblatex}
\addbibresource{sample.bib}
\usepackage{csquotes}
\usepackage{authblk}

\usepackage{booktabs}
\usepackage{tabularx}
\usepackage{graphicx}
\usepackage{float}
\usepackage{hyperref}
\usepackage{eurosym}
% \restylefloat{table}

% TODO no sé si algunos nombres se traducen
\title{Using Random Fourier Features with Random Forest}
\author{Albert Ribes}
% TODO poner la fecha adecuada
\date{Fecha de defensa}
\affil{Director: Lluís A. Belanche Muñoz}
\affil{Computer Science}
\affil{Grau en Enginyeria Informàtica}
\affil{Computació}
\affil{FACULTAT D’INFORMÀTICA DE BARCELONA (FIB)}
\affil{UNIVERSITAT POLITÈCNICA DE CATALUNYA (UPC) -- BarcelonaTech}
% Se deben incluir estos campos
% a) Títol
% b) Autor
% c) Data de defensa
% d) Director i Departament del Director
% e) Titulació
% f) Especialitat
% g) Centre: FACULTAT D’INFORMÀTICA DE BARCELONA (FIB)
% h) Universitat: UNIVERSITAT POLITÈCNICA DE CATALUNYA (UPC) – BarcelonaTech

\begin{document}
\maketitle
\tableofcontents
% \newpage

%%%%%%%%%%%%%%%%%%%%%%%%
%% Content starts here
%%%%%%%%%%%%%%%%%%%%%%%%


\section{Ideas generales}
El guión que me propuso LLuís es:
\begin{enumerate}
    \item Problema que ataco
    \item Por qué es importante
    \item Qué propongo en mi TFG
    \item Estado del arte en el problema que ataco
    \item Nociones generales del tema
    \begin{itemize}
        \item Machine Learning
        \item Árboles
        \item Logit
        \item RFF
        \item Nystroem
        \item Bootstrap
        \item Boosting
    \end{itemize}
    \item El trabajo propiamente dicho (explicar lo que voy a hacer)
    \item Experimentos
    \item Conclusiones y Trabajo futuro
    \item Referencias
    \item Apéndices
\end{enumerate}

\section{Introduction}
    \subsection{Problem to solve}
Todavía no se consigue suficiente precisión con el Machine Learning
    \subsection{Why it is important to solve this problem}
Con precisión más alta se podría aplicar el machine learning en otros campos
    \subsection{Project proposal}
Incrementar el accuracy que se puede conseguir con algunos problemas mezclando
la técnica del bagging (y quizá del boosting) con el tema los RFF

Actualmente el bagging solo se usa con Decision Tree porque es muy inestable.
Con lo que propongo aquí, podría ser factible usarlo con otros algoritmos más
estables

\section{Background}
Más o menos, cada uno debería ocupar entre media y una página
    \subsection{Machine Learning}
    \subsection{Classification and Regression}
    \subsection{Review de los principales modelos que existen}
        \subsubsection{Decision Tree}
        \subsubsection{Logistic Regression}
        \subsubsection{SVM}
    \subsection{Las técnicas ensembling}
        \subsubsection{Bagging}
\begin{itemize}
    \item Inventado por Leo Breiman
    \item Pretende reducir el sesgo
\end{itemize}
        \subsubsection{Boosting}
\begin{itemize}
    \item Adaboost (adaptive boosting)
    \item El siguiente estimador es más probable que contenga los elementos no
    no se han predicho bien en el anterior
    \item Se trata de ir modificando los pesos que tiene cada una de las instancias
    \item El entrenamiento de los modelos es secuencial, a diferencia del bagging
    \item Enterarme de quien lo inventó, y para qué ámbitos es útil
\end{itemize}
    \subsection{El bootstrap}
    \begin{itemize}
\item En bagging es bueno que los estimadores estén poco relacionados
entre ellos
\item Idealmente, usaríamos un dataset distinto para cada uno de los
estimadores, pero eso no siempre es posible
\item Una alternativa es usar un resampling con repetición sobre cada
uno de los estimadores para tener datasets un poco distintos entre ellos.
\item Enterarme de la cantidad de elementos distintos que se espera que queden
en el subconjunto, y quizá hablar de la cantidad de aleatoriedad
    \end{itemize}
    \subsection{Las funciones kernel}
    \subsection{Las Random Fourier Features}
    \subsection{Nystroem}
    \subsection{PCA}
    \subsection{Cross-validation}

\section{Workflow of the project}
  \subsection{La idea general un poco desarroyada}
  \subsection{State of the art con las RFF}



\section{Experimental results}
\section{Conclussion}
De momento, parece que algunos problemas sí que se benefician de esto, mientas
que otros no lo hacen
\section{Future work}
\begin{itemize}
\item El trabajo se ha centrado en problemas de clasificación, pero no hay
ningún motivo para que no se pueda aplicar el regresión. Se ha omitodo por
simplificar
\item Aquella teoría de que quizá se puede regular la cantidad de aleatoriedad
que añade el bootstrap, y quizá inventar un bootstrap con un parámetro para
regular la cantidad de aleatoriedad
\item Pensar en aquella teoría de que quizá se puede inventar un procedimiento
para, dato un problema determinado con sus datos, sacar un número que sea
representativo de la cantidad promedio de ruido que tiene. Puesto que quizá
es útil para este proyecto conocer la cantidad de aleatoriedad que tienen
los datos, para que se pueda regular
\end{itemize}
\section{Sustainability Report}
    % \section{Sustainability Report}
Consumo del equipo
CO2
Materiales Peligrosos
Materiales que vienen de zonas de conflicto
Qué sabemos de nuestros proveedores, y si la fabricación de la maquinaria se ha
hecho en una instalación sin riesgo, ni para las personas ni para la naturaleza

El impacto que tiene el proyecto directa e indirectamente en la gente
Ha sido diseñado pensando en cómo se ha de reciclar y reparar, siguiendo
conceptos de economía circular?

\begin{table}[]
\begin{tabular}{|l|l|l|l|}
\hline
                       & \textbf{Project development} & \textbf{Exploitation} & \textbf{Risks}      \\ \hline
\textbf{Environmental} & Consumption design           & Ecological Footprint  & Environmental risks \\ \hline
\textbf{Economic}      & Project bill                 & Viability plan        & Economic risks      \\ \hline
\textbf{Social}        & Personal impact              & Social impact         & Social risks        \\ \hline
\end{tabular}
\end{table}

\subsection{Environmental}
El impacto medio-ambiental de mi TFG
¿Cuál es el coste medio-ambiental de los productos TIC?
¿Cuantos recursos son necesarios para fabricar un dispositivo?
¿Cuánto consumen estos dispositivos durante su vida útil?
¿Cuantos resuduos se generan para fabricarlos?
¿Qué hacemos de los dispositivos una vez ha terminado su vida útil? ¿Los tiramos y
contaminamos el medio de nuestro entorno? ¿Los enviamos al tercer mundo, y
contaminamos el suyo?

La huella ecológica se puede medir, por ejemplo, en Kilovatios hora y en emisiones
de CO2

La huella ecológica que tendrá el proyecto durante toda su vida útil
¿Mi TFG contribuirá a reducir el consumo energético y la generación de resudios?
¿O los incrementará?




\subsection{Economic}
Costos de portarlo a terme i assegurar la seva pervivència

Estimar los recursos materiales y humanos necesarios para la realización del
proyecto. Sería como preparar la factura que se le pasará al cliente, teniendo
en cuenta que se terminará en un término bien definido

Una planificación temporal y
Una explicación de si he pensado algún proceso para reducir costes

Estimar las desviaciones que he tenido respecto a las planificaciones iniciales.

Para evaluarlo durante su vida útil, he de hacer un pequeño estudio de mercado,
y analizar en qué se diferencia de los ya existentes, si se mejora en algún
aspecto, o no.
Reflexionar sobre si mi producto tendrá mercado o no, y sobretodo, explicarlo

Además de estudiar los costes, plantearse si es posible reducirlos de alguna
manera

Plantear posibles escenarios que por razones de limitaciones en el tiempo y de
recursos no tendré en cuenta, pero que podrían perjudicar la viabilidad
económica del proyecto

Suele salir a unos 20000 \euro{} el TFG

\subsection{Social}

Implicaciones sociales, tanto sobre el colectivo al que se dirige el proyecto
como sobre otros colectivos

\subsubsection{Impacto personal}
En qué me ha afectado a mí, y a mi entorno cercano, la realización de este
proyecto. En qué me ha cambiado la vida, o si ha cambiado mi visión sobre el
tema.
¿Ha hecho que me de cuenta de situaciones que antes ignoraba?

\subsubsection{Impacto social}
Implicaciones que la realización de este proyecto puede tener sobre la sociedad.
Hay que identificar al colectivo de los afectados.
Los colectivos pueden ser: los propietarios, los gestores del proyecto, los
trabajadores, los proveedores, los consumidores (usuarios directos), o terceros
(usuarios indirectos o pasivos)
Puedo consultar los estándares del GRI

\subsubsection{Riesgos sociales}
Explicar posibles escenarios probables, pero no significativos, que no puedo
abordar por falta de tiempo o de recursos o de capacidad, y que podrían
perjudicar a personas relacionadas con mi proyecto


% Aquí empieza el índice que tenía en la fita de seguiment
% \section{Context}
%     \subsection{General Framework}
%     \subsection{Into the specifics}
%     \subsection{State of the Art}
%     \subsection{Problem to solve}
%
% \section{Planning}
%     \subsection{Original Planning}
%     \subsection{Problems encountered with original planning}
%     \subsection{Proposed new planning}
%
% \section{Methodology}
%     \subsection{Original Proposed Methodology}
%     \subsection{Problems encountered with original methodology}
%     \subsection{New methodology}
%
% \section{Alternatives Analysis}
%     \subsection{Language for development}
%     \subsection{Running environment}
%     \subsection{Machine Learning Algorithms}
%
% \section{Knowledge Integration}
%
% \section{Implication and Decision Making}
%     \subsection{Meetings with director}
%     \subsection{Goals achievement}
%     \subsection{Rigour in scientific procedures}
%
% \section{Laws and regulations}
%     \subsection{My responsibility}
%     \subsection{Others responsibility}
%     Esto es algo que hizo \cite{dirac} y también \cite{einstein}








\printbibliography
\end{document}
