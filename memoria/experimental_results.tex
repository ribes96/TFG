1. Plantear la hipótesis \\
2. Plantear los experimentos que habría que hacer \\
3. Mostrar los resultados que he obtenido \\
4. Discutir temas 2 a 2 \\
5. Contrastar las hipótesis, y ver si se han refutado o support \\




% Los experimentos que vamos a realizar son los siquientes
%   \subsubsection{De precisión}
%   \begin{description}
%     \item[Los modelos simples] Este nos permitirá ver cuál es aproximadamente
%     la dificultad de cada problema, para poder compararlo con las siguientes
%     modificaciones.
%     \item[Black Box vs. White Box] Siempre hemos tenido la teoría de que el
%     White Box será mejor, pero hay que comprobarlo.
%     \item[Bag vs Ensemble usando black box]
%     \item[Bag vs Ensemble usando white box]
%   \end{description}
%   \subsection{De tiempo}

\subsection{Enfrentar resultados 2 a 2}
  \begin{note}
    Los experimentos ya los tengo planteados 2 a 2. Entonces, se trata de
    comentar cada uno de los experimentos?
  \end{note}

\subsection{Contrastar hipótesis con resultados}
