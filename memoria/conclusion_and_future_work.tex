% \section{Conclussion}
De momento, parece que algunos problemas sí que se benefician de esto, mientas
que otros no lo hacen


\subsection*{Future work}

\begin{note}
  \begin{itemize}
    \item Problemas de regresión
    \item Aproximar kernels distintos a rbf
    \item Ver el comportamiento con problemas reales, que tienen missings,
    clases mal balanceadas, etc.
    \item Otros tipos de ensembles, como el boosting
    \item Generalización de bootstrap que regule la cantidad de aleatoriedad
    \item Procedimiento para sacar la cantidad promedio de ruido que tiene un
    problema
  \end{itemize}
\end{note}
\begin{itemize}
 \item El trabajo se ha centrado en problemas de clasificación, pero no hay
       ningún motivo para que no se pueda aplicar el regresión. Se ha omitodo por
       simplificar
 \item Aquella teoría de que quizá se puede regular la cantidad de aleatoriedad
       que añade el bootstrap, y quizá inventar un bootstrap con un parámetro para
       regular la cantidad de aleatoriedad
 \item Pensar en aquella teoría de que quizá se puede inventar un procedimiento
       para, dato un problema determinado con sus datos, sacar un número que sea
       representativo de la cantidad promedio de ruido que tiene. Puesto que quizá
       es útil para este proyecto conocer la cantidad de aleatoriedad que tienen
       los datos, para que se pueda regular
\end{itemize}
