% Tolo lo que ponga aquí son simplemente ideas, que al final habrá que quitar

\begin{note}

  En los procedimientos que usan liblinear, cambiar el parámetro de
  \textit{tolerancia}, y volver a poner la cantidad de iteraciones a
  su cantidad normal.

  La gamest que tengo implementada está mal, porque eleva al cuadrado. Hay
  que cambiarlo



  \section*{Ideas generales}

  El guión que me propuso LLuís es:
  \begin{enumerate}
   \item Problema que ataco
   \item Por qué es importante
   \item Qué propongo en mi TFG
   \item Estado del arte en el problema que ataco
   \item Nociones generales del tema
         \begin{itemize}
          \item Machine Learning
          \item Árboles
          \item Logit
          \item RFF
          \item Nystroem
          \item Bootstrap
          \item Boosting
         \end{itemize}
   \item El trabajo propiamente dicho (explicar lo que voy a hacer)
   \item Experimentos
   \item Conclusiones y Trabajo futuro
   \item Referencias
   \item Apéndices
  \end{enumerate}


  Parece que en algún paper se ha comprobado como no hay una diferencia
  significativa entre usar RFF y usar Nystroem

  El paper \cite{rff_equals_nys} se comprueba como no hay una especial diferencia

  En el paper \cite{nys_better_rff} se comprueba como Nystroem es mejor cuando
  hay un eigen-gap más grande

\end{note}
