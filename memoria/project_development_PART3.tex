\subsection{Datasets}

\begin{note}
  \begin{itemize}
    \item 8 datasets distintos
    \item Están normalizados
    \item Solo trabajo con variables numéricas. Además, los kernels con
    categóricas no tienen mucho sentido
    \item Las cosas particulares que he hecho con los datasets:
    \begin{itemize}
      \item Mezclar datos de train y test para luego hacer mi propia separación
      \item Cuando había poca presencia de una clase, hacer resampling con la
      misma cantidad
      \item No trabajar cosas como el skewness, y outliers
      \item Eliminar columnas en las que todo eran 0
      \item Reducir el conjunto de instancias
    \end{itemize}
  \end{itemize}
\end{note}

He enfocado el trabajo únicamente con problemas de clasificación. He hecho
pruebas con 8 datasets distintos.

Todos ellos los he normalizado a media 0 y varianza 1, y he usado dos tercios
para train y un tercio para test.

Por limitaciones de las implementaciones de los algoritmos que tengo, no podía
trabajar con variables categóricas, de modo que algunas de ellas he tenido que
convertirlas a float.

Algunos datasets me los daban separados en 2 subconjuntos, uno para train y el
otro para test. Yo los he mezclado, y ya si eso he hecho la separación,
aleatoria, por mi cuenta más tarde.

En algunos casos me daban muchísimas instancias, y yo no necesitaba tantas, y
por lo tanto he cogido un subconjunto

En algunos casos el dataset no estaba bien balanceado, había muchas instancias
de un tipo y pocas de otro. Yo he hecho un subconjunto para cada clase, todos
con la misma cantidad de instancias. Esto lo he hecho porque el objetivo de este
trabajo no tiene nada que ver con clases mal balanceadas.

% TODO convertir os subsection en subsection*

\subsubsection{Pen Digits}
\cite[See][]{pen-digits}

Distinguir entre los 10 dígitos (0-9) de un conjunto de imágenes. El dataset
se ha generado cogiendo las coordenadas $x$ e $y$ del trazo hecho por una
persona para dibujar ese número e interpolando 8 puntos normalmente espaciados
en todo el trazo del dibujo.

La tabla con la información es \ref{info-dts-pen-digits}

% \dtsInfo{pen-digits}{16}{10992}{10}

\subsubsection{Covertype}
\cite[See][]{covertype}

Identificar el tipo de terreno usando información como la elevation, el slope,
la horizontal distance to nearest surface water features, etc.

En el dataset original había un atributo con 40 columnas binarias que
identificaba el tipo de tierra, con la Soil Type Designation. Estas 40
columnas las he convertido a una sola variable, con números del 1 al 40.

La tabla con la información es \ref{info-dts-covertype}


% \dtsInfo{covertype}{12}{4900}{7}

\subsubsection{Satellite}
\cite[See][]{satellite}

La página de este dataset dice que son 7 clases, pero una de ellas no tiene
ninguna presencia, y por lo tanto yo no la he contado para nada.

Cada instancia es una parcela de $3 \times 3$ pixels y para cada pixel nos
dan 4 números, cada uno de ellos es el color que se ha captado con 4
different spectral bands. Por eso son 36: $9 \times 4 = 36$.

Lo que se predice es el tipo de terreno que contenía ese pixel, como plantación
de algodón,

La tabla con la información es \ref{info-dts-satellite}
% \dtsInfo{satellite}{36}{6435}{6}

\subsubsection{Vowel}
\cite[See][]{vowel}

% \vowel-info
% \vowelInfo

Me daban los datos separados por quien había dicho cada vocal, y yo lo he
mezclado todo.

Se trata de ver cual de las 11 vocales que tiene el idioma inglés es la que se
ha pronunciado. Para ello se usan 10 atributos.

La tabla con la información es \ref{info-dts-vowel}
% \dtsInfo{vowel}{10}{990}{11}

\subsubsection{Fall Detection}
\cite[See][]{fall-detection}

Se trata de identificar cuando una persona se ha caido al suelo o en qué otro
estado está (de pie, tumbado, caminando).

La tabla con la información es \ref{info-dts-fall-detection}
% \dtsInfo{fall-detection}{6}{16382}{6}

\subsubsection{MNIST}
\cite[See][]{mnist}

La tabla con la información es \ref{info-dts-mnist}
% \dtsInfo{mnist}{12}{4900}{7}

\subsubsection{Segment}
\cite[See][]{segment}

La tabla con la información es \ref{info-dts-segment}
% \dtsInfo{segment}{717}{5000}{10}

\subsubsection{Digits}
\cite[See][]{digits}

La tabla con la información es \ref{info-dts-digits}
% \dtsInfo{digits}{64}{5620}{10}


\allDtsInfo
