% \section{Sustainability Report}
Consumo del equipo
CO2
Materiales Peligrosos
Materiales que vienen de zonas de conflicto
Qué sabemos de nuestros proveedores, y si la fabricación de la maquinaria se ha
hecho en una instalación sin riesgo, ni para las personas ni para la naturaleza

El impacto que tiene el proyecto directa e indirectamente en la gente
Ha sido diseñado pensando en cómo se ha de reciclar y reparar, siguiendo
conceptos de economía circular?

\begin{table}[]
\begin{tabular}{|l|l|l|l|}
\hline
                       & \textbf{Project development} & \textbf{Exploitation} & \textbf{Risks}      \\ \hline
\textbf{Environmental} & Consumption design           & Ecological Footprint  & Environmental risks \\ \hline
\textbf{Economic}      & Project bill                 & Viability plan        & Economic risks      \\ \hline
\textbf{Social}        & Personal impact              & Social impact         & Social risks        \\ \hline
\end{tabular}
\end{table}

\subsection{Environmental}
El impacto medio-ambiental de mi TFG
¿Cuál es el coste medio-ambiental de los productos TIC?
¿Cuantos recursos son necesarios para fabricar un dispositivo?
¿Cuánto consumen estos dispositivos durante su vida útil?
¿Cuantos resuduos se generan para fabricarlos?
¿Qué hacemos de los dispositivos una vez ha terminado su vida útil? ¿Los tiramos y
contaminamos el medio de nuestro entorno? ¿Los enviamos al tercer mundo, y
contaminamos el suyo?

La huella ecológica se puede medir, por ejemplo, en Kilovatios hora y en emisiones
de CO2

La huella ecológica que tendrá el proyecto durante toda su vida útil
¿Mi TFG contribuirá a reducir el consumo energético y la generación de resudios?
¿O los incrementará?




\subsection{Economic}
Costos de portarlo a terme i assegurar la seva pervivència

Estimar los recursos materiales y humanos necesarios para la realización del
proyecto. Sería como preparar la factura que se le pasará al cliente, teniendo
en cuenta que se terminará en un término bien definido

Una planificación temporal y
Una explicación de si he pensado algún proceso para reducir costes

Estimar las desviaciones que he tenido respecto a las planificaciones iniciales.

Para evaluarlo durante su vida útil, he de hacer un pequeño estudio de mercado,
y analizar en qué se diferencia de los ya existentes, si se mejora en algún
aspecto, o no.
Reflexionar sobre si mi producto tendrá mercado o no, y sobretodo, explicarlo

Además de estudiar los costes, plantearse si es posible reducirlos de alguna
manera

Plantear posibles escenarios que por razones de limitaciones en el tiempo y de
recursos no tendré en cuenta, pero que podrían perjudicar la viabilidad
económica del proyecto

Suele salir a unos 20000 \euro{} el TFG

\subsection{Social}

Implicaciones sociales, tanto sobre el colectivo al que se dirige el proyecto
como sobre otros colectivos

\subsubsection{Impacto personal}
En qué me ha afectado a mí, y a mi entorno cercano, la realización de este
proyecto. En qué me ha cambiado la vida, o si ha cambiado mi visión sobre el
tema.
¿Ha hecho que me de cuenta de situaciones que antes ignoraba?

\subsubsection{Impacto social}
Implicaciones que la realización de este proyecto puede tener sobre la sociedad.
Hay que identificar al colectivo de los afectados.
Los colectivos pueden ser: los propietarios, los gestores del proyecto, los
trabajadores, los proveedores, los consumidores (usuarios directos), o terceros
(usuarios indirectos o pasivos)
Puedo consultar los estándares del GRI

\subsubsection{Riesgos sociales}
Explicar posibles escenarios probables, pero no significativos, que no puedo
abordar por falta de tiempo o de recursos o de capacidad, y que podrían
perjudicar a personas relacionadas con mi proyecto
