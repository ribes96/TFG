% \subsection{General Idea}
%   La idea general un poco desarroyada
%
%   Las funciones kernel son funciones que se pueden expresar de la forma:
%
%   \begin{equation}
%      \kappa(\vx, \vy) = \phi(\vx)^T\phi(\vy)
%   \end{equation}
%   Es decir, como producto escalar de una función de sus parámetros. Un kernel muy
%   popular es el RBF (\eng{Radial Basis Function}) gausiano, que es este:
%
%   \begin{equation}
%    \kappa({\vx, \vy}; \sigma) = e^{-\frac{\norm{\vx - \vy}}{2\sigma^2}}
%   \end{equation}
%
%   La función implícita $\phi$ ($\mathcal{L} \mapsto \mathcal{H}$) de este kernel
%   tiene una dimensionalidad infinita, y se sabe que para cualquier conjunto de
%   datos se puede encontrar un kernel RBF $\kappa$ tal que su función implícita
%   $\phi$ es capaz de separarlos mediante un híper-plano.
%
%   A pesar de que la función $\phi$ tiene dimensionalidad infinita (
%   $\mathcal{H} \equiv \reals^\infty$
%   ) es posible extraer una aproximación aleatoria de la misma con una precisión arbitraria,
%   mediante el uso de \eng{Random Fourier Features} \cite{rff} (RFF). Otra técnica
%   que también se puede utilizar es el método Nystroem. Con estos
%   métodos se puede extraer $\psi(\vx) \approx \phi(\vx)$ y usarlos para lo que
%   haga falta.
%
%   La extracción de estas aproximaciones se ha usado con anterioridad junto con
%   métodos de redes neuronales, y ha mostrado muy buenos resultados. En este
%   nosotros tratamos de usarlas con otros modelos. En particular, hemos
%   estudiado los modelos de Decision Tree, Logit y LinearSVC, en combinación con
%   varios tipos de ensemble.
%
%   El uso de ensembles está muy extendido junto con los Decision Tree. Esto se
%   debe a que éste es un modelo muy inestable, y una pequeña alteración en los
%   datos puede producir resultados muy distintos. Estas condiciones son idoneas
%   para hacer un comité de Decision Trees, entrenarlos con datos ligeramente
%   distintos y elegir la solución qué más árboles hayan predicho.
%
%   Pero este procedimiento no tiene ningún sentido hacerlo con modelos que no son
%   inestables. Si los modelos no son inestables, la mayoría de los estimadores
%   responderán la misma solución, y no servirá para nada haber entrenado tantos
%   modelos distintos. Es como si en un comité de expertos todos ellos opinaran
%   igual: para eso no necesitamos todo un comité, con un solo experto nos habría
%   bastado.
%
%   La técnica de \eng{bagging} utiliza el \eng{bootstrap} para generar datasets
%   distintos para cada uno de los estimadores. Consiste en hacer un remuestreo de
%   los datos con repetición para cada uno de los estimadores. Esta diferenciación
%   que se produce es suficiente para los Decision Tree, pero es demasiado leve con
%   los métodos más estables, como Logit y LinearSVC.
%
%   Pero los RFF y Nystroem abren una nueva puerta. Puesto que son aproximaciones
%   aleatorias de un conjunto infinito, podemos sacar tantos mapeos distintos
%   como queramos de los datos originales, y por lo tanto podemos diferenciar
%   todavía más los datasets generados para cada uno de los estimadores.
%
%   Además de todo esto, hay una ventaja adicional: entrenar una \eng{Support
%   Vector Machine} (SVM) con kernel lineal es más barato que entrenar una no lineal, por
%   ejemplo una que use RBF. Si usamos una SVM lineal, pero en vez de entrenarla
%   con los datos originales la entrenamos con los datos $\psi(\vx)$, tenemos un
%   coste similar al de entrenar una SVM lineal pero con una precisión equiparable
%   a una RBF-SVM. Esto ya se ha hecho antes.
%
%   Existen varias formas de combinar las RFF con los métodos ensembles. Básicamente,
%   hay dos parámetros que podemos elegir: qué tipo de ensemble usar y en qué
%   momento usar las RFF.
%
%   Cuando se combina un ensemble con los RFF, básicamente hay dos momentos en
%   los que se puede usar el mapeo. Un momento es nada más empezar, antes de
%   que el ensemble haya visto los datos, y el ensemble trabaja normalmente, solo
%   que en vez de recibir los datos originales recibe un mapeo de los mismos.
%   Este método se abstrae completamente de lo que hace el ensemble, y lo trata
%   como una caja negra. \eng{Black Box}.
%
%   El otro método consiste en usar el RFF, no nada más empezar y el mismo para
%   todos los estimadores, sino justo antes del estimador: se hace un mapeo nuevo
%   para cada uno de los estimadores. Este método ya se mete dentro de lo que es
%   un ensemble, y por tanto diremos que es de caja blanca (\eng{White Box}).
%
%   Pero se sabe que se obtienen mejores resultados cuando hay bastante diversidad
%   entre los estimadores del ensemble. Se nos presentan ahora dos formas de crear
%   diversidad en el ensemble. Una de ellas es la forma clásica, mediante el
%   bootstrap, que ha mostrado muy buenos resultados con el \eng{Decision Tree}.
%   Pero ahora podemos usar también la aleatoriedad de los RFF para generar
%   esa diversidad. Entonces tenemos dos opciones: usar los RFF ellos solos o usarlos
%   junto con el Bootstrap. A usarlos junto con el bootstrap le llamaremos un
%   \eng{Bagging}, mientras que si no usamos bootstrap le llamaemos un \eng{Ensemble}.
%
%   Tenemos entonces varias combinaciones entre manos:
%
%   \paragraph{Black Bag}
%   Black Box model con Bagging. Primero se hace un mapeo de los datos y después se
%   hace un bootstrap con ellos para cada uno de los estimadores. Si los estimadores
%   son \eng{Decision Tree} es los mismo que un \eng{Random Forest}, pero no con los
%   datos originales, sino con el mapeo.
%
%   \paragraph{White Bag}
%   White Box model con Bagging. Primero se hace un bootstrap de los datos, para
%   cada uno de los modelos, y después para cada uno de ellos se hace un mapeo de
%   los datos.
%
%   \paragraph{White Ensemble}
%   White Box model sin baging. Se hace un mapeo para cada uno de los estimadores,
%   todos ellos usando todos los datos originales.
%
%
%   El \eng{Black Ensemble} no tiene ningún sentido hacerlo, porque en ese caso
%   todos los estimadores recibirían exactamente los mismos datos, y por lo tanto
%   todos producirían exactamente los mismos resutados, a no ser que tuvieran algún
%   tipo de aleatoriedad, como los Decision Tree. A pesar de que tengamos ese caso
%   particular con los DT, no lo vamos a tratar.
%
%   Y luego, por supuesto, haremos pruebas con un modelo simple usando los RFF, sin
%   usar ningún tipo de ensemble.
% \subsection{Hyperparameters}
%
% Existen los siguiente híper-parámetros:
% \subsubsection*{Decision Tree}
% El algoritmo de Python de DT tiene muchísimos híperparámetros: la máxima
% profundidad, el mínimo de hojas, mínimo de instancias para hacer split,
% mínimo decrecimiento de la impureza de un nodo para hacer split, etc.
%
% Trabajar con todos ellos no era viable, de modo que únicamente hemos considerado
% el \eng{min\tu  impurity\tu decrease}. Todos los demás los hemos dejado que
% sobreajusten. De esta manera el modelo es más parecido al de R, y facilita hacer
% comparaciones
% \subsubsection*{Logit}
% Se puede hacer Logit con regularización de \textit{w} o sin hacerla, pero la
% implementación en Python que tenía disponible obligaba a hacerla. Como decidimos
% que no queríamos hacer regularización, hemos puesto un valor que hace que la
% regularización sea mínima.
%
% \subsubsection*{Support Vector Machine}
%
% Tiene un parámetro \textit{C} para regular la importancia que se le da a los
% vectores que quedan fuera del margen. Es el que hemos considerado.
%
% También hay otros híperparámetros:
%
% \begin{description}
%   \item[Cantidad  de features] Se puede sacar una cantidad arbitraria de
%   features, cuanto mayor mejor se aproxima al kernel. Después de muchas pruebas,
%   vimos que con 500 features ya no se puede mejorar mucho más, de modo que
%   hemos hecho todas las pruebas usando exactamente 500 features.
%   \item[Gammas] Tanto el RFF como el Nystroem tienen un parámetro \textit{gamma},
%   que es el del kernel que quieren simular, el RBF.
%   \item[Cantidad de estimadores de los ensembles] Se sabe que incrementear la
%   cantidad de estimadores no empeora la precisión del modelo, pero coger un
%   número demasiado alto incrementa el tiempo de entrenamiento. Hemos fijado
%   este parámeto a 500.
% \end{description}
%
% El procedimiento que hemos usado para encontrar los híperparámetros para los
% experimentos ha sido el siguiente:
%
% Hemos usado crossvalidation, pero hemos intentado que éste fuera solo de un
% híperparámetro para no incrementear demasiado el coste. Por lo tanto, hemos
% tenido que hacer algunas simplificaciones.
%
% En los experimentos que implican los modelos simples, sin ningún añadido, se
% ha hecho crossvalidation con el único parámetro que tienen: DT el min\tu impurity
% \tu decrease, Logit no tiene ninguno (pues hemos forzado la regulación al
% mínimo) y SVM tiene la C. Hemos hecho crossvalidation con esos.
%
% Cuando usamos algún sampler, también tenemos una gamma que encontrar. En esos
% casos, puede ser que estemos usando un modelo simple, o un ensemble de modelos.
% Cuando se trate de un modelo simple usaremos una gamma que sobreajuste, mientras
% que haremos crossvalidation con sus híper-parámetros.
%
% Cuando se hace un ensemble, hay que dejar que cada uno de los estimadores
% sobreajusten. Para ello usaremos unos híperparámetros que sobreajusten, mientras
% que haremos crossvalidation con la gamma.
%
% \subsection{Hypothesis}
%
% Poner la hipótesis 2 veces: la primera con muy poco detalle, sin palabras
% técnicas, en la introducción. Indicando que mas tarde se van a precisar
%
% La segunda será más técnica.
%
% \begin{itemize}
%   % \item Las RFF permiten que hacer un ensemble de Logit o de SVM tenga sentido
%   \item Con las RFF podemos mejorar la precisión de un Logit o de una SVM
%   haciendo un ensemble con ellos.
%   \item Es posible conseguir un rendimiento parecido al de una RBF-SVM con
%   un coste mucho menor usando las RFF
%   \item Bootstrap + RFF puede generar demasiada aleatoriedad en algunos
%   problemas, y perjudicar la precisión del algoritmo
%   % \item El modelo white box es capaz de conseguir más precisión que el black box
%   \item Por la naturaleza misma del DT, no se beneficiará demasiado de usar los
%   RFF. Estudiar si un método que no se basa en productos escalares se puede
%   beneficiar de usar Random Fourier Features. No hablar en particular del dt,
%   sino de los que no se basan en producto escalar.
%   \item Tenemos tres tipos de ensembles: Black Bag, Grey Bag y White Bag. El
%   White bag será el mejor.
% \end{itemize}


% \subsection{Datasets}
%
% He enfocado el trabajo únicamente con problemas de clasificación. He hecho
% pruebas con 8 datasets distintos.
%
% Todos ellos los he normalizado a media 0 y varianza 1, y he usado dos tercios
% para train y un tercio para test.
%
% Por limitaciones de las implementaciones de los algoritmos que tengo, no podía
% trabajar con variables categóricas, de modo que algunas de ellas he tenido que
% convertirlas a float.
%
% Algunos datasets me los daban separados en 2 subconjuntos, uno para train y el
% otro para test. Yo los he mezclado, y ya si eso he hecho la separación,
% aleatoria, por mi cuenta más tarde.
%
% En algunos casos me daban muchísimas instancias, y yo no necesitaba tantas, y
% por lo tanto he cogido un subconjunto
%
% En algunos casos el dataset no estaba bien balanceado, había muchas instancias
% de un tipo y pocas de otro. Yo he hecho un subconjunto para cada clase, todos
% con la misma cantidad de instancias. Esto lo he hecho porque el objetivo de este
% trabajo no tiene nada que ver con clases mal balanceadas.
%
% \subsubsection{Pen Digits}
% \cite[See][]{pen-digits}
%
% Distinguir entre los 10 dígitos (0-9) de un conjunto de imágenes. El dataset
% se ha generado cogiendo las coordenadas $x$ e $y$ del trazo hecho por una
% persona para dibujar ese número e interpolando 8 puntos normalmente espaciados
% en todo el trazo del dibujo.
%
% La tabla con la información es \ref{info-dts-pen-digits}
%
% \dtsInfo{pen-digits}{16}{10992}{10}
%
% \subsubsection{Covertype}
% \cite[See][]{covertype}
%
% Identificar el tipo de terreno usando información como la elevation, el slope,
% la horizontal distance to nearest surface water features, etc.
%
% En el dataset original había un atributo con 40 columnas binarias que
% identificaba el tipo de tierra, con la Soil Type Designation. Estas 40
% columnas las he convertido a una sola variable, con números del 1 al 40.
%
% La tabla con la información es \ref{info-dts-covertype}
%
%
% \dtsInfo{covertype}{12}{4900}{7}
%
% \subsubsection{Satellite}
% \cite[See][]{satellite}
%
% La página de este dataset dice que son 7 clases, pero una de ellas no tiene
% ninguna presencia, y por lo tanto yo no la he contado para nada.
%
% Cada instancia es una parcela de $3 \times 3$ pixels y para cada pixel nos
% dan 4 números, cada uno de ellos es el color que se ha captado con 4
% different spectral bands. Por eso son 36: $9 \times 4 = 36$.
%
% Lo que se predice es el tipo de terreno que contenía ese pixel, como plantación
% de algodón,
%
% La tabla con la información es \ref{info-dts-satellite}
% \dtsInfo{satellite}{36}{6435}{6}
%
% \subsubsection{Vowel}
% \cite[See][]{vowel}
%
% Me daban los datos separados por quien había dicho cada vocal, y yo lo he
% mezclado todo.
%
% Se trata de ver cual de las 11 vocales que tiene el idioma inglés es la que se
% ha pronunciado. Para ello se usan 10 atributos.
%
% La tabla con la información es \ref{info-dts-vowel}
% \dtsInfo{vowel}{10}{990}{11}
%
% \subsubsection{Fall Detection}
% \cite[See][]{fall-detection}
%
% Se trata de identificar cuando una persona se ha caido al suelo o en qué otro
% estado está (de pie, tumbado, caminando).
%
% La tabla con la información es \ref{info-dts-fall-detection}
% \dtsInfo{fall-detection}{6}{16382}{6}
%
% \subsubsection{MNIST}
% \cite[See][]{mnist}
%
% La tabla con la información es \ref{info-dts-mnist}
% \dtsInfo{mnist}{12}{4900}{7}
%
% \subsubsection{Segment}
% \cite[See][]{segment}
%
% La tabla con la información es \ref{info-dts-segment}
% \dtsInfo{segment}{717}{5000}{10}
%
% \subsubsection{Digits}
% \cite[See][]{digits}
%
% La tabla con la información es \ref{info-dts-digits}
% \dtsInfo{digits}{64}{5620}{10}
